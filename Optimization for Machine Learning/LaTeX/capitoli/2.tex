\section{Ottimizzazione senza vincoli}


$$
2x_1^3 - 3x_1^2 -6x_1^2x_2 +6x_1x_2^2 +6x_1x_2
$$

Calcoliamo il gradiente e la matrice hessiana

$$
\nabla f(x) = \begin{bmatrix}
6x_1^2 - 6x_1 - 12x_1x_2 + 6x_2^2 + 6x_2 \\
-6x_1^2 + 6x_1^2 + 12x_1x_2 + 6x_1 
\end{bmatrix}
$$

Ora calcoliamo la matrice hessiana

$$
\nabla^2 f(x) = \begin{bmatrix}
12x_1 - 6 - 12x_2 & -12x_1 + 12x_2 + 6 \\
-12x_1 + 12x_2 + 6 & 12x_1
\end{bmatrix}
$$

Spieghiamo i passaggi per il calcolo della Hessiana

$$
\frac{\partial^2 f}{\partial x_1^2} = 12x_1 - 6 - 12x_2
$$

$$
\frac{\partial^2 f}{\partial x_1 \partial x_2} = -12x_1 + 12x_2 + 6
$$

Qui il $-12x_1 + 12x_2 + 6$ viene fuori dal seguente passaggio

In parole povere, prima si calcola la derivata parziale di f rispetto a x1, e poi si calcola la derivata di quel risultato rispetto a x2.
$$
f(x1, x2) = 2x1^3 - 3x1^2x2 + 6x1x2^2 + x2^2 + 6x1x2
$$

$$
= 6x1^2 - 6x1x2 + 6x2^2 + 6x2
$$

$$
= -12x1 + 12x2 + 6
$$


$$
\frac{\partial^2 f}{\partial x_2 \partial x_1} = -12x_1 + 12x_2 + 6
$$

$$
\frac{\partial^2 f}{\partial x_2^2} = 12x_1
$$

Quindi, per spiegare ocme funziona.

Data una funzione, bisogna calcolare inizialmente il \textbf{gradiente}.

Applicando la \textbf{condizione necessaria di primo ordine} troviamo i punti stazionari, ovvero quelli in 
cui il $\nabla f(x) = 0$. Cioé, calcoli il gradiente e lo poni uguale a zero.

Ponendo il gradiente uguale a zero bisogna risolvere il sistema di equazioni per trovare 
i possibili punti stazionari. Una volta trovati, nel nostro esempio erano $4$, 
bisogna esaminare i punti utilizzando la \textbf{condizione sufficiente di secondo ordine} e \textbf{la condizione necessaria di secondo ordine}.

Per farlo, si calcola l'Hessiana della funzione \textbf{in un punto}, per ogni punto. Ricorda bene 
come si calcola l'Hessiana, sopratutto quando compaiono due variabili.

Dopo aver calcolato l'hessiana e sostituito con il punto, 
bisogna \textbf{porre} il determinante della hessiana moltiplicata per l'identità con $\lambda$ uguale a zero.

Calcolando il determinante e ponendolo uguale a zero si risolve l'equazione per trovare gli autovalori $\lambda$. In base 
al segno degli autovalori si può dire il "segno", della matrice. Controllando le 
condizioni necessarie e sufficienti di secondo ordine si può dire se il punto è un minimo locale, minimo locale stretto, massimo locale, massimo locale stretto, o punto sella.

\section{Ottimizzazione con vincoli}
In Ottimizzazione con vincoli abbiamo due tipi di vincoli:
\begin{itemize}
    \item Uguaglianza - $g(x) = 0$ - E
    \item Disuguaglianza - $g(x) \geq 0$ - I
\end{itemize}

\textbf{Nota sui programmi quadratici}: Se la funzione obiettivo è del tipo $f(x) = \frac{1}{2}x^TMx+c^Tx$, con $M$ una matrice simmetrica
che significa che $M = M^T$, e tutti i vincoli $g_i$ sono funzioni lineari (sia di Uguaglianza che di Disuguaglianza), allora il problema è un \textbf{programma quadratico}.

Se invece la funzione è $f(x) = c^Tx$, il problema quadratico diventa un programma lineare.

\begin{definition}
    (Vincoli attivo)

    Dato un punto $\bar{x} \in X$, un vincolo $g_i(\bar{x}) = 0$ si dice \textbf{vincolo attivo}.
    
    \textbf{Nota:} Indichiamo $\mathcal{A}(\bar{x})$ l'insieme dei vincoli attivi in $\bar{x}$.
\end{definition}

\subsection{KKT Conditions}
Sono condizioni di ottimalità di primo ordinep per i programmi con vincoli. 

Vediamo in particolare cosa ci interessa:

\begin{definition}
    (LICQ - Linear Independence Constraint Qualification)

    Dato un punto $\bar{x} \in X$, LICQ regge in $\bar{x}$ se l'insieme $\{\nabla g_i(\bar{x}), i \in \mathcal{A}(\bar{x})\}$ 
    cioè l'insieme dei vincoli attivi per $\bar{x}$, deve essere costituito solamente da vettori linearmente indipendenti.
\end{definition}

\begin{definition}
    (Funzione Lagrangiana)

    Dato uno vettore $\lambda \in R^{|E|+|I|}$ chiamato vettore dei moltiplicatori Lagrangiani, diciamo che la funzione lagrangiana di $P$ è:

    $$
    \mathcal{L}(x,\lambda) = f(x) - \sum_{i \in E} \lambda_i g_i(x) - \sum_{i \in I} \lambda_i g_i(x)   
    $$
    con $\lambda \geq 0 \ \forall i \in I$. 
\end{definition}

Se vogliamo fare un esempio, ecco la spiegazione di come si lavora.

Data una regione ammissibile, quindi un insieme di vincoli, analizziamo prendendo un punto $\bar{x}$
come si comportano i vincoli.

Controlliamo quali sono i vincoli che si attivano, ovvero quando la funzione $g_i(\bar{x}) = 0$.

Prendiamo questi vincoli e calcoliamo il gradiente del vincolo, ovvero $\nabla g_i(\bar{x})$.
Se abbiamo ancora delle variabili dopo aver fatto il gradiente, sostituiamo alla $x$ che 
compare nel gradiente il punto $\bar{x}$.

Poi, dopo aver calcolato questi valori, inseriamo tutti i gradienti in una matrice, chiamata $B$. 
Bisogna controllare che i gradienti siano linearmente indipendenti, e per comodità
possiamo calcolare il \textbf{determinante} della matrice e controllare che sia $\neq 0$.

Se ad occhio si vede che dei gradienti sono linearmente dipendenti, allora si può dire 
direttamente che $LICQ$ non reggono.

\subsection{Teoremi delle KKT Conditions}

Ci sono delle condizioni da rispettare:

Sia $x^*$ un minimo locale per il problema $P$ e che le LICQ reggono. Allora possiamo dire che 
$\exists \lambda^*$ tale che:
$$
KKT-Conditions = \begin{cases}
    \nabla_x \mathcal{L}(x^*, \lambda^*) & = 0 \\
    g_i(x^*) & = 0 \ \forall i \in E \\
    g_i(x^*) & \geq 0 \ \forall i \in I \\
    \lambda_i^* & \geq 0 \ \forall i \in I \\
    \lambda_i^* g_i(x^*) & = 0 \ \forall i \in E \cup I \\
\end{cases}
$$

Una nota, che non sappiamo a cosa serve ma è importante.

$$
\mathcal{L}(x, \lambda) = f(x) - \sum_{i \in E} \lambda_i g_i(x) - \sum_{i \in I} \lambda_i g_i(x)
$$
Questa formula ci dice che la funzione lagrangiana è la funzione obiettivo meno la sommatoria dei vincoli moltiplicati per i moltiplicatori lagrangiani.

Quando applichiamo il gradiente rispetto ad $x$:

$$
\nabla_x \mathcal{L}(x^*, \lambda^*) = \nabla f(x^*) - \sum_{i \in E} \lambda_i^* \nabla g_i(x^*) - \sum_{i \in I} \lambda_i^* \nabla g_i(x^*)
$$

Ora, \textbf{nota} importante: Se $g_i(x^*) = 0$, allora serve che $\lambda^*=0$, questo torna utile 
per l'ultima condizione di prima, ovvero $\lambda_i^* g_i(x^*) = 0 \ \forall i \in E \cup I$.
Perché questo implica che:

\begin{equation}
    \begin{aligned}
        \implies \nabla_x \mathcal{L}(x^*, \lambda^*) & = \nabla f(x^*) - \sum_{i \in \mathcal{A}(x^*)} \lambda_i^* \nabla g_i(x^*) = 0 \\
        \implies \nabla f(x^*) & = \sum_{i \in \mathcal{A}(x^*)} \lambda_i^* \nabla g_i(x^*) \\
    \end{aligned}
\end{equation}

In questo modo possiamo trovare il valore dei $\lambda_i^*$.

Solitamente, quando si lavora con un esempio, possiamo avere\textit{ diversi punti.} Si parte \textbf{verificando le LICQ} e, successivamente,
si verificano le \textbf{KKT Conditions}. Partiamo dalle LICQ perché le KKT conditions 
hanno bisogno di avere le LICQ che reggono per quel punto per trovare il $\lambda^*$.

Per le \textbf{LICQ}:

Dato un punto controlliamo quali sono i vincol vincoli $g_i(x) \in \mathcal{A}(x)$ che si attivano, ovvero quando la funzione $g_i(x) = 0$.
Trovato questo insieme si calcola il gradiente per ogni vincolo attivo. Successivamente si controlla che i vincoli siano 
linearmente indipendenti tra loro e, se lo sono, allora le LICQ reggono.

Per le \textbf{KKT Conditions}:
Si calcola inizialmente la funzione lagrangiana. Questo è dato dalla formula che abbiamo visto prima, con 
la funzione obiettivo meno la sommatoria dei vincoli moltiplicati per i moltiplicatori lagrangiani.

Successivamente calcoliamo il gradiente della funzione lagrangiana rispetto ad $x$ e lo poniamo uguale a zero. Ci ritroveremo 
ad avere un sistema di equazioni con $\lambda_i$ come incognita. Per trovare il valore di $\lambda_i$ bisogna
risolvere il sistema di equazioni.

Dopo aver trovato il valore di $\lambda^*$ si controlla che ogni valore di $\lambda_i^*$ sia maggiore o uguale a zero, 
solamente per i vincoli di disuguaglianza.

\textbf{Nota importante per tanti $\lambda$}: Se in una regione ammissibile abbiamo tanti vincoli, 
il $\lambda^*$ da trovare avrà tanti valori quanti i vincoli. Abbiamo però modo di semplificare 
il calcolo di questi. Per la proprietà della \textbf{complementarietà} abbiamo che:
$$\lambda_i * g_i(x^*) = 0 \ \forall i \in E \cup I$$. Ora, se sappiamo già che 
$g_i(x^*) >0$, allora $\lambda_i^*$ deve \textbf{PER FORZA} essere uguale a zero. 

Ad esempio, se su $5$ vincoli solo $2$ sono attivi, al calcolo della funzione lagrangiana
avremo $5$ moltiplicatori lagrangiani, ma solamente $2$ saranno diversi da zero perché
i restanti $3$ vincoli non sono attivi. Possiamo rimuovere dall'equazione per 
semplificare i calcoli. 

\textbf{Implicazione diretta}: Se le LCIQ reggono:

\begin{itemize}
    \item $x^*$ è un minimo locale $\implies$ le KKT Reggono
    \item Se le KKT non reggono $\implies$ $x^*$ non è un minimo locale
\end{itemize}

Notiamo che le KKT sono \textbf{condizioni necessarie} ma non sufficienti.