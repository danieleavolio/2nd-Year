\section{Stable Matching}
\begin{definition}
    (Matching)

    Matching riguarda agenti che hanno preferenze su altri agenti. Questo si 
    applica a tante situazioni, per esempio:
    \begin{itemize}
        \item Dottori match negli ospedali
        \item Studenti match nelle università
        \item Persone alle task
        \item Membri del team 
        \item Kidney exchange
    \end{itemize}
\end{definition}

\textbf{Matching a 2 lati}: Ci sono 2 gruppi i agenti che hanno preferenze 
sui membri dell'altro gruppo. L'obiettivo è di trovare un matching stabile.

\subsection{Problema del matrimonio stabile}

Dati $n$ uomini e $n$ donne. Ogni agente ha una preferenza lineare 
sull'altro sesso. L'obiettivo è:

\textbf{Matching stabile} tra uomini e donne. Nessun uomo vuole divorziare
il partner assegnato. Non possiamo ri-assegnare una coppia per 
migliorare la situazione di una coppia senza peggiorare quella di un'altra.

\begin{definition}
    (Gale-Shapley Algorithm)

    Esiste un matching stabile in ogni combinazione di preferenze tra uomini e donne.

    Il funzionamento è il seguente:
    \begin{enumerate}
        \item In ogni round ogni uomo che non è fidanzato si propone alla donna che favorisce tra le donne a cui non si ancora proposta
        \item In ogni donna, ogni donna sceglie il suo favorito tra i suoi pretendenti e quello a cui è legata
    \end{enumerate}
\end{definition}

Un esempio dell'algoritmo:

\begin{equation}
    \begin{aligned}
        A & = [X,Y,Z]\\
        B & = [Z,X,Y]\\
        C & = [Z,Y,C]\\
        X & = [A,B,C]\\
        Y & = [B,C,A]\\
        Z & = [A,C,B]\\
    \end{aligned}
\end{equation}

Il flow è il seguente:

\begin{enumerate}
    \item A si propona a X
    \item B si propone a Z
    \item C si propone a Z
    \begin{itemize}
        \item Z accetta C perché lo preferisce a B
        \item Z rifiuta B e rimane con C
    \end{itemize}
    \item B si propone a X
    \begin{itemize}
        \item X rifiuta B e rimane con A
    \end{itemize}
    \item B si propone a Y
\end{enumerate}

L'algoritmo termina con un matching stabile. Diciamo anche delle piccole proprietà:

\begin{itemize}
    \item Termina sempre 
    \item Ritorna sempre un matching stabile
    \item Ha una complessità $quadratica$ : $O(n^2)$
\end{itemize}

\begin{definition}
    M-Optimal e W-Optimanl Matchings

    Un matching stable viene chiamato \textbf{M-Optimal} se ogni uomo è contento almeno quanto in ogni altro matching stabile.

    Un matching stabile viene chiamato \textbf{W-Optimal} se ogni donna è contenta almeno quanto in ogni altro matching stabile.
\end{definition}

Il matching di Gale-Shapley è M-Optimal e non W-Optimal. Perché? Perché
gli uomini si propongono alle donne e quindi hanno la possibilità di
scegliere. Le donne invece non hanno questa possibilità.

\newpage