\section{Appunti VCG}

Assumiamo
\begin{itemize}
\item m oggetti
\item n acquirenti
\end{itemize}

Gli acquirenti puntano sugli oggetti e possono fare più di 1 puntata.

Possono puntare per:
\begin{itemize}
\item 1
\item 2
\item 3
\item 1,2
\item 1,3
\item 2,3
\item 1,2,3
\end{itemize}

Acquirenti = \{A,B,C\}

Oggetti = \{a,b\}

\[
\begin{aligned}
v(A) &= \{a:2, b:2, a,b:5\}\\
v(B) &= \{a:5, b:0, a,b:0\}\\
v(C) &= \{a:3, b:2, a,b:6\}
\end{aligned}
\]

Vogliamo massimizzare la social welfare.

In questo caso: B $\leftarrow$ a, C $\leftarrow$ b  

La somma della social welfare è 7.

Calcolare questo è NP-Hard. Un po' un Problema dello zaino sotto steroidi.

Nel caso del Problema 4, per stabilire la social welfare, prendiamo il path con la social welfare massimizzata, quindi il maximized weighted path.

Ora, problema della pagamento degli acquirenti.

B dice che a vale 5, C dice che b vale 2.

Ma in questo caso, dobbiamo usare le aste a secondo prezzo.

Se usassimo le aste a primo prezzo gli acquirenti potrebbero mentire sul prezzo che vogliono pagare.

Con le Aste a secondo prezzo gli acquirenti sono incentivati a dire la verità. Anche Ebay usa le aste a secondo prezzo.

Quindi, in questo caso, dobbiamo assegnare il pagamento in base al prezzo del secondo acquirente. Gli acquirenti, ribassando il prezzo, non possono ottenere un guadagno maggiore.

Quindi, in questo caso, ogni scommettitore paga le esternalità che impone agli altri.

Ci sono 2 valori:
\begin{itemize}
\item Valore massimo di welfare senza giocatore $i$ nell'asta
\item Valore della welfare degli altri giocatori, con $i$ nell'asta
\end{itemize}

\[
\begin{aligned}
p_A &= 7 - 7 = 0 \\
p_B &= 6 - 2 = 4 \\
p_C &= 6 - 5 = 1
\end{aligned}
\]

Per il calcolo di $p_B$, vediamo che la welfare sarebbe 6 senza $B$ nell'asta. Mentre il valore 2 viene fuori da $C$ che ha detto che b vale 2. Quindi, $p_B = 6 - 2 = 4$.

Per il calcolo di $p_C$, vediamo che la welfare sarebbe 6 senza $C$ nell'asta. Mentre il valore 5 viene fuori da $B$ che ha detto che a vale 5. Quindi, $p_C = 6 - 5 = 1$.

Ora, in formule, si risolve con queste formule:

Allocazione Ottima $\omega^*$:

\[
\arg \max_{\omega \in \Omega} \sum_{i \in N} v_i(\omega)
\]

Ogni giocatore $i$ paga: $p_i(\omega^*)$ = 

\[
\max_{\omega \in \Omega} \sum_{j \neq i} v_j(\omega^*) - \sum_{j \neq i \in N} v_j(\omega^*)
\]

Nel caso del problema 4.

Nel caso del nostro problema, abbiamo che la path welfare massimo:

[2,4,1,3] = 100

Quanto devo pagare 1?

\begin{itemize}
\item Rimuovo 1 dal grafo. Quanto vale la path massima senza 1? [2,4] = 60
\item Quanto vale la path massima con 1 ma senza contare il suo contributo? [2,4,3] = 90
\end{itemize}

Ok, quindi, in questo caso, 1 deve pagare -30\$, cioe' sono io che devo pagare 30\$ a 1.

\begin{enumerate}
    \item Prendo il path con welfare massima
    \begin{itemize}
        \item Lo chiamo $\omega^*$
    \end{itemize}
    \item Prendo il path con welfare massima senza il giocatore $i$
    \begin{itemize}
        \item Lo chiamo $\omega^*_i$
    \end{itemize}
    \item Calcolo la differenza tra:
    \begin{itemize}
        \item $\omega^*_i$
        \item $\omega^* - i*10$
    \end{itemize}
    \item $\omega^*_i - (\omega^* - i*10)$
\end{enumerate}

