\section{Cos'è la Game Theory | Teoria dei giochi}
\label{sec:cos_e_la_game_theory_teoria_dei_giochi}

La \textbf{teoria dei giochi} è una disciplina che studia il comportamento
decisionale multi-persona, usato per fare predizioni su come \textbf{agenti
    razionali multipli} interagiscono o si comportano in situazioni di
\textit{cooperazione} o in situazione di \textit{conflitto}.

Alcune definizioni di termini:
%use 
\begin{itemize}
    \item \textbf{Conflitto}: le azioni dei giocatori hanno effetto sugli Algorithmic
    \item \textbf{Cooperazione}: I giocatori possono collaborare per raggiungere un obiettivo
    \item \textbf{Comportamento razionale}: I giocatori vogliono massimizzare la loro \textit{utilità attesa | expected utility}
    \item \textbf{Predizione}: Il nostro obiettivo è sapere cosa faranno i giocatori, utilizzando \textit{solution concepts - concetti di soluzione}
\end{itemize}

\subsection{E cosa significa Algorithmic Game Theory?}
\label{sub:e_cosa_significa_algorithmic_game_theory}

Possiamo dire che algorithmic game theory è un punto d'incontro tra
\textbf{game theory} e \textbf{algorithm design} che punta a
\textit{\textbf{progettare algoritmi} che permettono deglle strategie in
    specifici ambienti}.

\subsection{Coalition Games}
\label{sub:coalition_game_theory}

La \textbf{coalition game theory} è una branca della game theory che studia le
interazioni tra gruppi di giocatori, che \textbf{collaborano} per
\textit{raggiungere un obiettivo comune}.

\textbf{Nota - Shapley Values:} Il concetto di \textbf{Shapley Values} è un concetto che permette di \textit{spiegare}, circa,
come un algoritmo di \textbf{machine learning} ha preso una decisione. Ad esempio, mostra
le \textit{feature} che hanno avuto un impatto maggiore nella decisione finale della predizione. In pratica mostra i vari
\textit{Join | Coalizioni} di features.

\textbf{Quali sono le domande più importanti in questa sezione?}
\begin{itemize}
    \item Quale coalizione è più probabile che venga formata?
    \item In che modo i giocatori devono dividere il premio? \textit{(Payoff)}
\end{itemize}

\subsection{Non-Cooperative Games}
\label{sub:non_cooperative_games}

In questo tipo di giochi, i giocatori \textbf{non hanno coalizioni} o comunque
non ne hanno bisogno.

Alcuni giochi che fanno parte di questa categoria:
\begin{itemize}
    \item Scacchi
    \item Sasso-Carta-Forbice
    \item \textit{Il dilemma del prigioniero}
\end{itemize}

\begin{figure}[H]
    \begin{center}
        \begin{tabular}{c|cc}
            \multicolumn{1}{c}{} & \multicolumn{2}{c}{Giocatore 1}            \\
            \cline{2-3}
            Giocatore 2          & Collabora                       & Tradisci \\
            \cline{1-3}
            Collabora            & (-1,-1)                         & (-5,0)   \\
            Tradisci             & (0,-5)                          & (-3,-3)  \\
            \cline{1-3}
        \end{tabular}
    \end{center}
    \caption{Esempio di dilemma del prigioniero}
\end{figure}

In questo gioco, la \textbf{strategia} migliore per il singolo è quella di
\textbf{tradire} l'altro giocatore, in quanto è quella che massimizza la sua
utilità, precisamente andrebbe a \textbf{perdere 0 punti}, mentre l'altro
giocatore ne perderebbe 5.

\subsection{Tree Decomposition}
\label{sub:tree_decomposition}

Alcuni problemi sui grafi hanno una complessità di \textbf{NP-HARD} su dei
grafi arbitrari, e hanno bisogno di alcune soluzioni che avranno
implementazioni complesse e \textbf{programmazione dinamica.}

\subsection{Computational Social Choice}
\label{sub:computational_social_choice}

Questa sezione parla di computazione di risultati risultanti da \textbf{regole
    di voto | voting rules} e quali problemi ci possono essere nel rappresentare le
preferenze dei giocatori. \
\begin{figure}[H]
    \begin{center}
        \begin{tabular}{c|ccc}
            \multicolumn{1}{c}{} & \multicolumn{3}{c}{Verdetto}                         \\
            \cline{2-4}
                                 & Evidenza1                    & Evidenza2 & Colpevole \\
            \cline{1-4}
            Giudice1             & 1                            & 0         & Innocente \\
            Giudice2             & 0                            & 1         & Innocente \\
            Giudice3             & 1                            & 1         & Colpevole \\
            \cline{1-4}
        \end{tabular}
    \end{center}
    \caption{Esempio di votazione}
\end{figure}

\begin{quote}
    \textit{Maggiore è il numero di persone che votano, maggiore è la probabilità che
        il risultato sia corretto.}
\end{quote}

\subsection{Mechanism Design}

E' un tipo di \textbf{reverse game theory}. Invece di analizzare come i
giocatori si comportano in un gioco, lo scopo del \textit{mechanism design} è
quello di \textbf{creare un gioco} per portare i giocatori a
\textbf{\textit{comportarsi in un modo specifico}} che \textit{vogliamo noi.}
Un esempio molto semplice è il \textit{maccanismo di asta di Ebay}. Altro
esempio è quello dei \textit{carrelli dei supermecati.} Il fatto di dover
utilizzare una moneta per utilizzare il carrello \textbf{porta la persona} a
dover riportare il carrello nello stesso posto, invece di lasciarlo in un luogo
qualsiasi del supermercato.

\subsection{Fair Division of Indivisible Goods}
\label{sub:fair_division_of_indivisible_goods}

In questa sezione si parla di come dividere delle risorse in modo \textbf{fair}
tra i giocatori. Ok?

\subsection{Cake Cutting}
\label{sub:cake_cutting}

In questa sezione si parla di come dividere dei \textbf{beni continui} in bsae alle \textit{preferenze dei giocatori}.
\begin{enumerate}
    \item Fairness
    \item Proportionality
    \item Envy-freeness
\end{enumerate}