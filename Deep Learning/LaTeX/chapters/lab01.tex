\section{Lab: Introduzione Python}
\subsection{MatPlotLib}

\subsubsection{Plots}

\begin{lstlisting}[language=Python]
import matplotlib.pyplot as plt
import numpy as np

x = np.linspace(0, 10, 100)
plt.plot(x, np.sin(x))
plt.show()
\end{lstlisting}

Non c'è molto da dire, il codice è autoesplicativo. La funzione \texttt{plot}
prende in input due array, uno per l'asse delle ascisse e uno per l'asse delle
ordinate. In questo caso, \texttt{x} è un array di 100 punti equidistanti tra 0
e 10, mentre \texttt{np.sin(x)} è un array di 100 punti che rappresentano il
seno dei punti di \texttt{x}. La funzione \texttt{show} mostra il grafico.

%fai la figure che è descritta nel codice

\begin{tikzpicture}
    \begin{axis}[
            axis lines = left,
            xlabel = $x$,
            ylabel = {$f(x)$},
        ]
        %Below the red parabola is defined
        \addplot [
            domain=0:10,
            samples=100,
            color=red,
        ]
        {sin(deg(x))};
        \addlegendentry{$\sin(x)$}
    \end{axis}
\end{tikzpicture}

\subsubsection{Sub-plots}

\begin{lstlisting}[language=Python]
import matplotlib.pyplot as plt
import numpy as np

x = np.linspace(0, 10, 100)
plt.subplot(2, 1, 1)
plt.plot(x, np.sin(x))
plt.subplot(2, 1, 2)
plt.plot(x, np.cos(x))
plt.show()

\end{lstlisting}

La funzione \texttt{subplot} prende in input tre parametri: il numero di righe,
il numero di colonne e l'indice del subplot corrente. Nel caso di questo
esempio, il subplot corrente è il primo, quindi viene mostrato il grafico del
seno. Poi viene mostrato il secondo subplot, che è quello del coseno.

\begin{tikzpicture}
    \begin{axis}[
            axis lines = left,
            xlabel = $x$,
            ylabel = {$f(x)$},
        ]
        %Below the red parabola is defined
        \addplot [
            domain=0:10,
            samples=100,
            color=red,
        ]
        {sin(deg(x))};
        \addlegendentry{$\sin(x)$}
    \end{axis}
\end{tikzpicture}

\begin{tikzpicture}
    \begin{axis}[
            axis lines = left,
            xlabel = $x$,
            ylabel = {$f(x)$},
        ]
        %Below the red parabola is defined
        \addplot [
            domain=0:10,
            samples=100,
            color=red,
        ]
        {cos(deg(x))};
        \addlegendentry{$\cos(x)$}
    \end{axis}
\end{tikzpicture}

C'è anche qui poco da dire, il codice è autoesplicativo.

\subsection{NumPy}

La libreria NumPy è una libreria per Python che permette di lavorare con
array multidimensionali. Per importare la libreria, basta scrivere
\texttt{import numpy as np}.

Mi rompo le palle in maniera assurda di scrivere tutti gli esempi. Quindi questo capitolo 
penso sia abbastanza inutile.


E' possibile trovare il codice di \textbf{numPy} a questo link: \url{www.ciao.it}