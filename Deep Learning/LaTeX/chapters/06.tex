\section{Reti Neurali Convoluzionali Pre-allenate}

\textbf{Motivazione principale:} E' quello di ridurre il modello e riutilizzarlo in altri contesti, senza dover effettuare ancora un allenamento della rete.

Prendiamo l'esempio di \textbf{VCG-16} che è una rete allenata su un dataset di
immagini di una vecchia competizione. La cosa da capire è che quando vogliamo
utilizzare la rete \textbf{bisogna preparare l'input} per essere inserito nella
rete. Ad esempio, le immagini sono in $224x224$.

\begin{lstlisting}[language=Python]
# prebuild model with pre-trained weights on imagenet
model = VGG16(weights='imagenet'
, include_top=True)
model.compile(optimizer='sgd'
, loss='categorical_crossentropy')

# resize into VGG16 trained images' format
# add a dummy iniziatl axis
im = cv2.resize(img, (224, 224))
plt.imshow(im)
im = np.expand_dims(im, axis=0)
im.astype(np.float32)
print(im.shape)
\end{lstlisting}

\begin{domanda}(Come usiamo la rete?)
\end{domanda}

Se volessimo, ad esempio, classificare \textit{modelli di laptop}, questa rete
può tornare utile nella classificazione? Come facciamo? La risposta \textbf{non
    è banale.}

La risposta è \textbf{si}. Il nucleo della questione è quello di utilizzare
\textbf{il primo layer} della rete già allenata. Questa è l'idea principale del
\textbf{transfer learning}. La motivazione è quella che i livelli dopo i primi
sono troppo specifici per il nostro scopo, mentre i primi layer sono più
generici e sono quellli che si occupano di estrarre le features.

\subsection{Come si usa il transfer learning?}

\begin{definition}(Feature extraction)
\end{definition}
Il concetto principale è quello di \textbf{utilizzare i primi layer} della rete,
prendendo solamente l'output che si ottiene ,circa, a metà della rete. Questo
significa che per ogni immagine del nostro dataset si estraggono le
\textbf{features} e le si salvano, creando cosi un dataset alternativo formato
dalle features.

Successivamente si utilizzano queste feature nel quale viene fatto
l'allenamento, e quindi il task sarà quello di \textbf{classificare le
    features} nella task che ci interessa.

\begin{definition}(Fine Tuning)
\end{definition}

Il concetto di \textbf{fine tuning} è quello di utilizzare i primi layer della
rete regolarmente, come se fossero sempre funzionanti, e successivamente
utilizzare una nuova rete collegata con gli ultimi nodi utilizzati fino a quel
punto e allenarla con il nostro dataset. In questo caso diciamo che
\textbf{congeliamo} i primi layer della rete.

Per unire un modello con un altro, in Keras è molto semplice perché si possono
trattare i modelli come layer.

\begin{lstlisting}

    model = VGG16(weights='imagenet', include_top=True)

    new_model = Sequential()
    new_model.add(model)
    new_model.add(Dense(2, activation='softmax'))

    new_model.compile(optimizer='sgd', loss='categorical_crossentropy')

\end{lstlisting}
In Keras, un modello può essere congelato impostando il parametro
\textbf{trainable} su \textbf{False}.

\begin{lstlisting}[language=Python]
    model.trainable = False
\end{lstlisting}

\newpage

\section{Oltre il modello sequenziale}

Il modello sequenziale in Keras è un'implementazione semplice e intuitiva di
una rete neurale, ma ha dei limiti in termini di flessibilità e complessità. In
particolare, il modello sequenziale è limitato a reti neurali feedforward,
ovvero reti neurali in cui l'informazione fluisce in una sola direzione, senza
cicli o connessioni ricorrenti.

Ci sono molti problemi di deep learning che richiedono reti neurali più
complesse, come ad esempio le reti neurali convoluzionali per l'elaborazione di
immagini o le reti neurali ricorrenti per l'elaborazione di sequenze. In questi
casi, il modello sequenziale può essere troppo limitato per modellare
efficacemente i dati.

Inoltre, il modello sequenziale non supporta la condivisione di pesi tra i
layer, il che può essere un'importante tecnica di regolarizzazione e può
aiutare a ridurre il numero di parametri del modello. Infine, il modello
sequenziale non supporta la definizione di grafi di calcolo arbitrari, il che
può essere necessario per alcune applicazioni avanzate di deep learning.

Un'alternativa è quella del \textbf{multi input}.

\subsection{Multi input e multi output}

Quando abbiamo più input, non possiamo utilizzare il modello sequenziale. Ad
esempio, abbiamo dati eterogenei che hanno bisogno di essere elaborati in modi
diversi tra loro. Ci possono essere vari approcci che permettono di gestire
questi casi:

\begin{itemize}
    \item Il merging dei moduli, quando più input che sono eterogenei
    \item Concatenate, quando gli input sono omogenei
    \item Multiple Output: In questo caso abbiamo un solo input ma più output, e quindi
          ci saranno regressori diversi in base alla feature da predire.
    \item Inception: L'architettura a inception è una rete neurale convoluzionale
          profonda che è stata introdotta per la prima volta nel 2014. L'idea principale
          è quella di utilizzare filtri di dimensioni diverse per estrarre le features e
          poi \textbf{concatenare} le features estratte per avere un output finale.
    \item Residual: E' un archiettura che sfrutta il concetto di \textbf{residual
              learning}, ovvero l'idea di aggiungere un layer che è un'identità rispetto
          all'input.
\end{itemize}

\subsection{Functional API}

Se ci pensiamo, un modello è un qualcosa che \textit{prende in input qualcosa}
e ritorna fuori \textit{un output}. Concettualmente, è la stessa cosa di una
\textbf{funzione}. Possiamo, quindi, considerare dei moduli come delle
funzioni.

\begin{equation}
    \begin{aligned}
        \mathcal{O}_1 = M_1(I_1)                        \\
        \mathcal{O}_2 = M_2(I_2)                        \\
        \mathcal{O}_3 = Q(\mathcal{O}_1, \mathcal{O}_2) \\
    \end{aligned}
\end{equation}

dove $Q$ è una funzione che prende in input due moduli e ritorna un output.

Vediamo un esempio di come funzionano e come usare queste \textit{functional
    API}

\begin{esempio}(Functional API)
\end{esempio}

\begin{lstlisting}[language=Python]

#Vogliamo avere questa rete 
# (input: 784-dimensional vectors)
# [Dense (64 units, relu activation)]
# [Dense (64 units, relu activation)]
# [Dense (10 units, softmax activation)]
# (output: logits of a probability distribution over 10 classes)
    
inputs = keras.Input(shape=(784,))

# Qui, invece di sequenzializzare i layer, li colleghiamo
dense = layers.Dense(64, activation="relu")
x = dense(inputs)

#Abbiamo passato l'input a questo layer e otteniamo x come output
#Aggiungiamo altri layer alla rete

x = layers.Dense(64, activation="relu")(x)
outputs = layers.Dense(10)(x)

#Possiamo specificare, ora, il modello 
model = keras.Model(inputs=inputs, outputs=outputs, name="Modello funzionale")
\end{lstlisting}

\textbf{Nota:} In questo momento stiamo solamende creando la struttura
della rete, ma non stiamo ancora definendo i pesi.

\newpage

\section{Adanvced Keras}

\subsection{Subclassing}

\begin{definition}(Subclassing)
\end{definition}

Il subclassing in Keras è una tecnica avanzata per la creazione di modelli di
deep learning personalizzati. Consiste nell'estendere la classe
"tf.keras.Model" e definire il modello all'interno del metodo "init" e il
passaggio in avanti all'interno del metodo "call". In questo modo, è possibile
creare modelli di deep learning altamente personalizzati e flessibili, con la
possibilità di definire qualsiasi tipo di struttura di rete e di utilizzare
qualsiasi tipo di operazione di calcolo.

In soldoni, il subclassing ci permette di definire:
\begin{itemize}
    \item Loss personalizzate
    \item Layer personalizzati
    \item Metriche personalizzate
    \item Modelli personalizzati
\end{itemize}

Chiaramente questo aumenta la complessità del codice, ma aumenta la
flessibilità e la possibilità di avere un controllo maggiore di quello che ci
fornisce di base Keras.

Ad esempio nelle \textbf{Generative Adversarial Networks - GAN} il concetto di
training viene fatto in modo diverso, poiché non si usa la \textbf{stochastic
    gradient descent}. Nelle GAN anche l'ordine di apprendimento delle reti è
importante; allenare contemporaneamente potrebbe portare a risultati strani e
non convenienti. Ci sono tecniche e scenari specifici che hanno apprendimenti
anch'essi specifici. Ovviamente, come detto prima, aumenta la complessità ma
aumenta anche la flessibilità.


\textbf{HA FATTO UN BORDELLO DI ROBA DALLA PAGINA DI TENSOR FLOW QUINDI BOH GUARDA La}


\newpage

