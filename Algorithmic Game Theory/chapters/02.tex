\section{Giochi di Coalizione e Concetti di Soluzione}
\label{sec:coalition}

Spesso la \textbf{teoria dei giochi} fa riferimento a tipologie di giochi in
cui gli agenti \textbf{non collaborano}, ma \textbf{competono} tra loro. In
questi casi si parla di \textbf{gioco non cooperativo}.

Parliamo di un ambiente in cui degli agenti interagiscono tra loro, e ogni
agente ha un suo \textbf{obiettivo} da raggiungere.

\begin{definition}[Gioco non cooperativo]
    Un gioco non cooperativo è un gioco in cui gli agenti non collaborano tra loro.
    \begin{itemize}
        \item Un set di agenti $N = \{1, \dots, n\}$.
        \item Ogni agente $i \in N$ ha un set di azioni $S$
        \item Ogni agnete $i \in N$ ha una funzione di utilità $u_i: S_1 \times S_2 \cdots
                  S_n \rightarrow \mathcal{R}$
    \end{itemize}
\end{definition}

\begin{definition}[Gioco cooperativo]
    Il contrario di giochi non cooperativi sono, banalmente, \textbf{i giochi cooperatvi}, in cui gli agenti \textbf{collaborano} tra loro.
\end{definition}

\textbf{Domanda:} In quale caso le coalizioni appaiono nella teoria dei giochi cooperativi?
\begin{itemize}
    \item Allocazioni di task
    \item Allocazione di risorse
    \item Esperienza degli agenti complementare tra loro
\end{itemize}

\textbf{Esempio di gioco cooperativo}: Immaginiamo di avere 9 agenti. Ora, gli agenti devono sceliere:
\begin{itemize}
    \item Con chi allearsi
    \item Come agire
    \item Come dividere il premio
\end{itemize}

Immaginiamo di avere $p_1, p_2, p_3, c_1,c_2,c_3,e_1,e_2,e_3$. Immaginiamo
queste 3 coalizioni:
\begin{itemize}
    \item $C_1 \{c_1,e_3,p_3\}$
    \item $C_2 \{c_3,e_2,p_2\}$
    \item $C_3 \{c_2,e_1,p_1\}$
\end{itemize}
Una \textbf{struttura di coalizione} è del tipo: $CS = <C_1, C_2, C_3>$.

Definiamo anche \textbf{il vettore azioni} $a = <a_{c_1}, a_{c_2}, a_{c_3}>$.

Allora possiamo avere un'\textbf{allocazione di risorse}
\begin{equation}
    u(C_3|a_{c_3})=30 \implies Allocazione:<p_1=12,c_2=3,e_1=15>
\end{equation}

Quindi, diciamo che nei giochi collaborativo:
\begin{itemize}
    \item I giocatori \textbf{formano coalizioni}
    \item Ogni coalizione ha associato un \textbf{worth}
    \item Alla fine c'è un \textbf{total worth} da distribuire
\end{itemize}
\subsubsection{Tassonomia dei giochi cooperativi}

Quando parliamo di giochi cooperativi parliamo di giochi in cui i giocatori tra
loro collaborano, fanno azioni insiee, e si formano dei vincoli tra loro. Ma
dobbiamo differenziare due tipi di \textbf{utility games}

\begin{definition}[Transferable Utility Games]
    La paga viene data al gruppo e si divide tra loro.
\end{definition}

\begin{definition}[Non Transferable Utility Games ]
    L'azione del gruppo fornisce la paga ai singoli giocatori in modo individuale.
\end{definition}

\textbf{Esempio di Transferable Utility Games}:

Hai N bambini, ognuno dei quali ha una certa quantità di denaro: il bambino
$i$-esimo ha $b_i$ dollari.

Sono in vendita tre tipi di vaschette di gelato:
\begin{itemize}
    \item Tipo 1 costa \$7 e contiene 500g.
    \item Tipo 2 costa \$9 e contiene 750g.
    \item Tipo 3 costa \$11 e contiene 1kg.
\end{itemize}

I bambini hanno una preferenza per il gelato e non si preoccupano del denaro.

Il risultato ottenuto da ciascun gruppo è la quantità massima di gelato che i
membri del gruppo possono acquistare unendo il loro denaro. Il gelato può
essere condiviso liberamente all'interno del gruppo.

\textbf{Formalizzazione dei giochi cooperativi:}

Un gioco di utilità trasferibile è una coppia $(N, v)$, dove:
\begin{itemize}
    \item $N = \{1, \ldots, n\}$ è l'insieme dei giocatori (anche chiamato coalizione grandiosa).
    \item $v: 2^N \rightarrow \mathbb{R}$ è la funzione caratteristica.
    \item Per ogni sottoinsieme di giocatori $C$, $v(C)$ è l'importo che i membri di $C$
          possono guadagnare lavorando insieme.
\end{itemize}

Facciamo delle assunzioni. Solitamente diciamo che $v$ è \textbf{normalizzato},
cioé $v(\emptyset) = 0$. Ci sono altri due casoi però:
\begin{itemize}
    \item \textbf{Non-negativo}: $v(C) \geq 0$ per ogni $C \subseteq N$.
    \item \textbf{Monotono}: $v(C) \leq v(D)$ per ogni $C,D$ t.c $C \subseteq D$.
\end{itemize}

Tutto questo non è sempre uguale e dipende sempre dallo scenario.

\textbf{Esempio del gioco dl gelato}: Abbiamo tre giocatori:
\begin{enumerate}
    \item C con 6 $€$
    \item M con 4 $€$
    \item P con 4 $€$
\end{enumerate}

Ora, abbiamo 3 tipi di gelato con :
\begin{itemize}
    \item Gelato 1: w = 500 e p = 7 $€$
    \item Gelato 2: w = 750 e p = 9 $€$
    \item Gelato 3: w = 1000 e p = 11 $€$
\end{itemize}

Cosa possiamo dire? Innanzitutto, \textbf{nessuno può comprare niente da solo.}
Quindi, se vogliamo che qualcuno compri qualcosa, dobbiamo formare una
coalizione. Le domande da fare sono: \textit{Quali azioni dobbiamo compiere? In
    quale modo ci dividiamo il premio?}.

\begin{align*}
    v(\emptyset)   & = v(\{C\}) = v(\{M\}) = v(\{P\}) = 0 \\
    v(\{C, M\})    & = 750                                \\
    v(\{C, P\})    & = 750                                \\
    v(\{M, P\})    & = 500                                \\
    v(\{C, M, P\}) & = 1000
\end{align*}

\begin{definition}[Outcome]
    Un outcome (o risultato) di un gioco di utilità trasferibile $G = (N, v)$ è una coppia $(CS, x)$, in cui:
    \begin{itemize}
        \item $CS = (C_1, \ldots, C_k)$ è una struttura di coalizione, cioè una partizione di $N$.
        \item $\bigcup_i C_i = N$, $C_i \cap C_j = \emptyset$ per $i \neq j$.
        \item $x = (x_1, \ldots, x_n)$ è un vettore di pagamento che distribuisce il valore di ciascuna coalizione in $CS$.
        \item $\sum_{i \in C} x_i = v(C)$ per ogni $C$ in $CS$ (Efficienza).
    \end{itemize}

    Supponiamo che $v(\{1, 2, 3\}) = 9$ e $v(\{4, 5\}) = 4$.

    Quindi, $(({1, 2, 3}, {4, 5}), (3, 3, 3, 3, 1))$ è un \textbf{risultato}.

    Invece, $(({1, 2, 3}, {4, 5}), (2, 3, 2, 3, 3))$ non è un risultato.

    I \textbf{trasferimenti tra coalizioni} non sono consentiti. Un risultato $(CS,
        \underbar{x})$ è chiamato \textbf{imputazione} se soddisfa la
    \textbf{razionalità individuale}: $x_i \geq v(\{i\})$ per tutti $i \in N$.

\end{definition}

\begin{definition}[Giochi superaddittivi]
    Un gioco di utilità trasferibile $G = (N, v)$ è chiamato superadditivo se $v(C \cup D) \geq v(C) + v(D)$ per qualsiasi due coalizioni disgiunte $C$ e $D$.

    Esempio: $v(C) = |C|^2$; $v(C \cup D) = (|C| + |D|)^2 \geq |C|^2 + |D|^2 = v(C)
        + v(D)$.

    Nei \textbf{giochi superadditivi}, due coalizioni possono sempre fondersi senza
    perdere denaro; quindi, possiamo assumere che i giocatori formino la coalizione
    grandiosa.

    Praticamente, quando due coalizioni collaborando ottengono un risultato che è
    almeno pari alla somma dei risultati che avrebbero ottenuto se avessero agito
    da sole.

    Un \textit{esempio} è il seguente:

    \begin{align*}
        v(\emptyset)   & = v(\{C\}) = v(\{M\}) = v(\{P\}) = 0 \\
        v(\{C, M\})    & = 750                                \\
        v(\{C, P\})    & = 750                                \\
        v(\{M, P\})    & = 500                                \\
        v(\{C, M, P\}) & = 1000
    \end{align*}

    In questo caso si vede che è un gioco superadditivo perché la collaborazione
    porta ad un risultato migliore.
\end{definition}

\subsection{Concetti di soluzione}

Assumiamo che la grande coalizione \textbf{N} sia formata. Quindi:
\[x = (x_1, x_2, \dots, c_n)\]
\[x_i \geq 0 \forall i \in N\]
\[x_1 + x_2 + \dots x_n = v(N)\]

Considera il gioco del gelato con la seguente funzione caratteristica. Si
tratta di un gioco superadditivo in cui gli esiti sono vettori di pagamento
(modi per dividere 1000).

Come dovrebbero i giocatori condividere il gelato?

Se lo dividono come (200, 200, 600), Charlie e Marcie possono ottenere più
gelato acquistando una vaschetta da 750g da soli e dividendo equamente. L'esito
(200, 200, 600) \textbf{non è stabile}!

\begin{definition}[Core O Nucleo]
    Il nucleo o core di un gioco è l'insieme di tutti gli \textbf{esiti che sono stabili}, cioé quegli
    esiti che no vengono scartati da nessuna coalizione.
    \[
        core(G) = \{(CS,\underbar{x}) | \sum_{i \in C} x_i \geq v(C) \ for \ any \ C \subseteq N\}\]
\end{definition}

\textbf{Nota}: $\textbf{x(c)}$ si identifica come la somma dei valori $\sum_{i \in C} x_i$.

\textit{Possiamo accorciare} in:

\begin{itemize}
    \item  $x \in R^n$ è un nucleo se $x(c) \geq v(c) \forall  C \subseteq N$ \\
    \item x(N) = v(N)
\end{itemize}

Torniamo al nostro esempio. Ora vediamo il concetto di \textbf{nucleo}
applicato

\begin{itemize}
    \item (200, 200, 600) \textbf{non è} nel nucleo:
    \item $v(\{C, M\}) > x_C + x_M$
    \item (500, 250, 250) \textbf{è nel} nucleo:
\end{itemize}

\textit{
    nessun sottogruppo di giocatori può deviare in modo che ciascun membro del sottogruppo ottenga di più.
}
Un vettore $(x_C, x_M, x_P)$ è nel nucleo se e solo se soddisfa le seguenti condizioni:
\begin{itemize}
    \item $x_C + x_M \geq v(\{C, M\})$ (stabilità)
    \item $x_C + x_P \geq v(\{C, P\})$ (stabilità)
    \item $x_P + x_M \geq v(\{P, M\})$ (stabilità)
    \item $x_C \geq v(\{C\})$ (razionalità individuale)
    \item $x_P \geq v(\{P\})$ (razionalità individuale)
    \item $x_M \geq v(\{M\})$ (razionalità individuale)
    \item $x_C + x_P + x_M = v(\{C, M, P\})$ (efficienza)
\end{itemize}

Queste \textbf{3 proprietà} sono importanti.

\begin{definition}[Giochi con nucleo vuoto]
    Il concetto di nucleo è molto attraente come concetto di soluzione. Purtroppo, ci sono alcuni giochi che
    hanno \textbf{un nucleo vuoto}.

    Consideriamo il gioco $G = (\{1, 2, 3\}, v)$ con la seguente funzione
    caratteristica $v$:

    $$
        v(C) =
        \begin{cases}
            1 & \text{se } |C| > 1 \\
            0 & \text{altrimenti}
        \end{cases}
    $$
\end{definition}

Considera un risultato $(CS, \underbar{x})$:

Supponi che \textbf{$CS = (\{1\}, \{2\}, \{3\})$.}
\begin{itemize}
    \item In questo caso, la coalizione grandiosa può deviare.
    \item $x_1 + x_2 + x_3 = v(\{1\}) + v(\{2\}) + v(\{3\}) < v(\{1, 2, 3\})$.
\end{itemize}

Cioè, \textbf{non è stabile}. Guarda la disequazione.

Supponi che \textbf{$CS = (\{1,2\}, \{3\})$.}
\begin{itemize}
    \item In questo caso, o 1 o 2 ottengono meno di 1, quindi possono deviare con 3.
    \item $x_1 + x_3 = x_1 + 0 < 1 < v(\{1, 3\})$.
\end{itemize}

Collaborando devono \textbf{dividere} questo \textbf{1} che ottengono. Quindi,
1 o 2 prenderò meno dell'altro e l'altro potrebbe deviare con il 3.

Supponi che \textbf{$CS = (\{1,2,3\})$}.
\begin{itemize}
    \item In questo caso, $x_i > 0$ vale per alcuni $i$, diciamo $i=3$.
    \item Quindi, $x(\{1,2\}) < 1$, ma $v(\{1,2\}) = 1$.
\end{itemize}\textbf{}

Quindi in ogni caso \textit{qualcuno potrebbe cercare una coalizione migliore
    di quella in cui si trova attualmente}. Questo porta ad avere un \textbf{nucleo
    vuoto}.

\subsubsection{Cosa fare quando il nucleo è vuoto?}

Questa situazione prende il nome di $\epsilon-Core$. In questo casi si vuole
approssimare un esito stabile.

Bisogna \textit{rilassare} il concetto di nucleo:

\begin{itemize}
    \item Nucleo: $(CS, x): x(C) \geq v(C)$ per ogni $C \subseteq N$
    \item \(\epsilon\)-nucleo: $(CS, x): x(C) \geq v(C) - \epsilon$ per ogni $C \subseteq N$
\end{itemize}

Solitamente questa nozione è definita solo per giochi superadditivi.

Per esempio, consideriamo il gioco \(G = (\{1, 2, 3\}, v)\), con \(v(C) = 1\)
se \(|C| > 1\), \(v(C) = 0\) altrimenti:
\begin{itemize}
    \item Il \(\frac{1}{3}\)-nucleo non è vuoto: \((\frac{1}{3}, \frac{1}{3},
          \frac{1}{3}) \in \frac{1}{3}\)-nucleo
    \item Il \(\epsilon\)-nucleo è vuoto per qualsiasi \(\epsilon < \frac{1}{3}\):
    \item \begin{itemize}
              \item \(x_i \geq \frac{1}{3}\) per qualche \(i = 1, 2, 3\); quindi \(x(N\{i\}) \leq \frac{2}{3}\), ma \(v(N\{i\}) = 1\).
          \end{itemize}
\end{itemize}

\begin{definition}[Nucleo minimo]

    Definiamo $\epsilon^*(G)$ come $\inf \{\epsilon | \epsilon$-core di $G$ non è
    vuoto$\}$.
    \begin{itemize}
        \item Si può dimostrare che $\epsilon^*(G)$-core non è vuoto.
    \end{itemize}
    La definizione di $\epsilon^*(G)$-core è il nucleo minimo di $G$.
    \begin{itemize}
        \item $\epsilon^*(G)$ è chiamato il valore del nucleo minimo.
    \end{itemize}
\end{definition}

Nel contesto del gioco $G = (\{1, 2, 3\}, v)$ con la funzione caratteristica
$v(C)$ definita come segue:
\begin{itemize}
    \item $v(C) = 1$ se $|C| > 1$
    \item $v(C) = 0$ altrimenti
\end{itemize}

\begin{itemize}
    \item Il 1/3-core non è vuoto: $(1/3, 1/3, 1/3) \in 1/3$-core
    \item Il $\epsilon$-core è vuoto per qualsiasi $\epsilon < 1/3$:
          \begin{itemize}
              \item $x_i \geq 1/3$ per qualche $i = 1, 2, 3$, quindi $x(N\{i\}) \leq 2/3$, ma $v(N\{i\}) = 1$.
          \end{itemize}
\end{itemize}

\subsection{Concetti di soluzione avanzati}
Ci sono in particolare due che sono molto importanti: \textbf{shapy value} e il
\textbf{Nucleolus}

\textbf{Più sofisticate considerazioni sulla stabilità}
\begin{itemize}
    \item \textbf{Nucleolus}: Il nucleolus è un concetto utilizzato nella teoria dei giochi cooperativi per valutare la giustizia nella distribuzione dei guadagni tra i giocatori.
    \item \textbf{Bargaining set}: L'insieme di contrattazione è un concetto che si riferisce agli insiemi di risultati in cui i giocatori trovano equo e ragionevole partecipare, dati i poteri di contrattazione.
    \item \textbf{Kernel}: Il nucleo è un sottoinsieme del nucleo in cui i giocatori non possono migliorare il proprio risultato cooperando in modo diverso.
\end{itemize}

\textbf{Concetto di fairness}
\begin{itemize}
    \item \textbf{Shapley value}: Il valore di Shapley è una soluzione per assegnare un valore a ciascun giocatore in modo equo, tenendo conto del loro contributo marginale a ogni possibile coalizione.
    \item \textbf{Banzhaf index}: L'indice di Banzhaf è una misura del potere di voto di ciascun giocatore in un gioco di voto ponderato.
\end{itemize}

\subsubsection{Nucleolus}

Definiamo ora il concetto di \textit{eccesso}

\begin{definition}[Eccesso]
    E' una misura che indica quanto la coalizione \textbf{è insoddisfatta.}

    \begin{equation}
        e(S,x) = v(S) - x(S)
    \end{equation}
\end{definition}

Facciamo un esempio numerico:

\begin{itemize}
    \item v(\{1\}) = v(\{2\}) = v(\{3\}) = 0
    \item v(\{1,2\}) = v(\{1,3\}) = v(\{2,3\}) = 1
    \item v(\{1,2,3\}) = 3
\end{itemize}

Minimizzare la insoddisfazione di ogni possibile coalizione. Applichiamo

\begin{itemize}
    \item x = (0,0,3) $\implies e(\{1,2\}, x) = v(\{1,2\}) - (x_1+x_2) = 1-0=1$
    \item x = (1,2,0) $\implies e(\{1,2\}, x) = v(\{1,2\}) - (x_1+x_2) = 1-3=-2$
\end{itemize}

Ritorniamo alla definizione di Nucleolus.

\textbf{Definizione da Schmeidler}: Il nucleolus $\mathcal{N}(\mathcal{G})$ di un gioco $\mathcal{G}$ è l'insieme:

\[
    \mathcal{N}(\mathcal{G}) = \{x \in \mathcal{X}(\mathcal{G})|\nexists y \in \mathcal{X}(\mathcal{G}) \ t.c\ \theta(y) \prec  \theta(X) \}
\]

Che significa proprio che non esiste un altro vettore di pagamento che è
preferito da tutti i giocatori rispetto a $x$. Quindi, il nucleolus è un
concetto di soluzione che è \textbf{stabile}.

\subsection{Shapley Value}

\textbf{Stability vs Fairness}

Consideriamo il gioco $G = (\{1, 2\}, v)$ con le seguenti caratteristiche:
\begin{itemize}
    \item $v(\emptyset) = 0$
    \item $v(\{1\}) = v(\{2\}) = 5$
    \item $v(\{1, 2\}) = 20$
\end{itemize}

Nel nucleo del gioco, abbiamo l'allocazione $(15, 5)$. In altre parole, il
giocatore 1 riceve 15 e il giocatore 2 riceve 5. È importante notare che il
nucleo rappresenta una situazione in cui nessun giocatore può ottenere un
risultato migliore deviando unilateralmente. In questo caso, il giocatore 2 non
può ottenere un risultato migliore deviando.

La domanda principale è: l'allocazione (15, 5) è equa? La giustizia in un
contesto di gioco può essere soggettiva e dipendere dalle aspettative e dagli
accordi tra i giocatori. Quindi, se l'allocazione è considerata equa o meno
potrebbe variare in base al contesto e alle aspettative dei giocatori.

No! Poiché 1 e 2 sono risultati simmetrici nel core possono essere ingiusti!
Come facciamo a dividere i pagamenti in modo equo?

\textit{Pensiamo a questo.} Un risultato equo dovrebbe premiare ciascun agente in base al loro contributo.
Nel primo tentativo, dato un gioco $G = (N, v)$, impostiamo $x_i = v(\{1,
    \ldots, i-1, i\}) - v(\{1, \ldots, i-1\})$. In altre parole, il pagamento per
ciascun giocatore è il loro contributo marginale alla coalizione dei loro
predecessori. Otteniamo $x_1 + \ldots + x_n = v(N)$; $x$ è un vettore di
pagamento.

Tuttavia, questo metodo non funziona poiché il pagamento di ciascun giocatore
dipende dall'ordine. Ad esempio, consideriamo il gioco $G = (\{1, 2\}, v)$, con
$v(\emptyset) = 0$, $v(\{1\}) = v(\{2\}) = 5$, e $v(\{1, 2\}) = 20$. In questo
caso, $x_1 = v(1) - v(\emptyset) = 5$ e $x_2 = v(\{1, 2\}) - v(\{1\}) = 15$.

Notiamo che i risultati non sono gli stessi, indipendentemente dall'ordine.
Pertanto, questa formulazione non produce risultati equi.

\textbf{Un'idea} per eliminare la dipendenza dall'ordine è quella di calcolare una media su tutte le possibili permutazioni degli ordini di arrivo.

Ad esempio, consideriamo il gioco $G = (\{1, 2\}, v)$, con $v(\emptyset) = 0$,
$v(\{1\}) = v(\{2\}) = 5$, e $v(\{1, 2\}) = 20$. Iniziamo calcolando i
pagamenti per due ordini diversi:
\begin{itemize}
    \item Per l'ordine (1, 2): $x_1 = v(1) - v(\emptyset) = 5$ e $x_2 = v(\{1, 2\}) -
              v(\{1\}) = 15$.
    \item Per l'ordine (2, 1): $y_2 = v(2) - v(\emptyset) = 5$ e $y_1 = v(\{1, 2\}) -
              v(\{2\}) = 15$.
\end{itemize}

Ora, calcoliamo i pagamenti mediando tra i due ordini:
\begin{itemize}
    \item $z_1 = (x_1 + y_1) / 2 = (5 + 15) / 2 = 10$
    \item $z_2 = (x_2 + y_2) / 2 = (15 + 5) / 2 = 10$
\end{itemize}

Il risultato ottenuto è equo, poiché ciascun giocatore riceve 10,
indipendentemente dall'ordine in cui sono considerati.

Una permutazione di $\{1, \ldots, n\}$ è una corrispondenza uno a uno da $\{1,
    \ldots, n\}$ a se stessa.

Denotiamo con $P(N)$ l'insieme di tutte le permutazioni di $N$.

Denotiamo con $S_{\pi}(i)$ l'insieme dei predecessori di $i$ in una
permutazione $\pi \in P(N)$.

Per $C \subseteq N$, definiamo $\delta_i(C) = v(C \cup \{i\}) - v(C)$ come il
contributo marginale del giocatore $i$ a $C$.

Il valore di Shapley del giocatore $i$ in un gioco $G = (N, v)$ con $|N| = n$ è
dato da:
\[
    \varphi_i(G) = \frac{1}{n!} \sum_{\pi: \pi \in P(N)} \delta_i(S_{\pi}(i))
\]

Supponiamo di scegliere una permutazione dei giocatori in modo uniforme e
casuale tra tutte le possibili permutazioni di $N$. In questo contesto, il
valore di Shapley $\varphi_i$ rappresenta il contributo marginale atteso del
giocatore $i$ alla coalizione dei suoi predecessori.

\subsubsection{Proprietà del valore di Shapley}
\begin{definition}[Proprietà 1]

    In qualsiasi gioco $G$, la somma dei valori di Shapley di tutti i giocatori,
    ossia $\varphi_1 + \ldots + \varphi_n$, è uguale al valore totale del gioco
    $v(N)$.

\end{definition}

\begin{definition}[Proprietà 2]

    Un giocatore $i$ è definito un giocatore fittizio (dummy) se $v(C) = v(C \cup
        \{i\})$ per qualsiasi insieme $C \subseteq N$.

    \textbf{Proposizione:} Se un giocatore $i$ è un giocatore fittizio, allora il suo valore di Shapley $\varphi_i$ è uguale a 0.

\end{definition}

\begin{definition}[Proprietà 3]

    Due giocatori $i$ e $j$ sono definiti simmetrici se $v(C \cup \{i\}) = v(C \cup
        \{j\})$ per qualsiasi insieme $C \subseteq N \setminus \{i, j\}$.

    \textbf{Proposizione:} Se $i$ e $j$ sono giocatori simmetrici, allora i loro valori di Shapley $\varphi_i$ e $\varphi_j$ sono uguali.

\end{definition}

\begin{definition}[Proprietà 4]

    Siano $G1 = (N, u)$ e $G2 = (N, v)$ due giochi con lo stesso insieme di
    giocatori. Allora $G = G1 + G2$ è il gioco con l'insieme di giocatori $N$ e la
    funzione caratteristica $w$ definita come $w(C) = u(C) + v(C)$ per tutti gli
    insiemi $C \subseteq N$.

    \textbf{Proposizione:} Il valore di Shapley di un giocatore $i$ nel gioco $G1 + G2$ è uguale alla somma dei valori di Shapley di $i$ nei giochi $G1$ e $G2$, ossia $\varphi_i(G1 + G2) = \varphi_i(G1) + \varphi_i(G2)$.

\end{definition}

\textbf{Proprietà riassunte:}
\begin{enumerate}
    \item \textbf{Efficienza:} $\varphi_1 + \ldots + \varphi_n = v(N)$
    \item \textbf{Dummy:} Se $i$ è un giocatore fittizio, allora $\varphi_i = 0$
    \item \textbf{Simmetria:} Se $i$ e $j$ sono giocatori simmetrici, allora $\varphi_i = \varphi_j$
    \item \textbf{Additività:} $\varphi_i(G1+G2) = \varphi_i(G1) + \varphi_i(G2)$
\end{enumerate}

\textbf{Teorema:} Il valore di Shapley è l'unico schema di distribuzione dei pagamenti che soddisfa le proprietà 1-4.

E' possibile scrivere la formula anche in questo modo:
\[
    \phi(i, \nu) = \sum_{C \subseteq N} \frac{(|N|-|C|)! \times (|C| -1)!}{|N|!}(\nu(C) - \nu(C\setminus \{i\}))
\]

