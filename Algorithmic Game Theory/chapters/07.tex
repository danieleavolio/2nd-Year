\section{Social Choiche}
\label{sec:social-choice}

Partiamo dando qualche definizione generale.

Il risultato e' fissato per ogni agente. Una volta presa una decisione, ogni
agente deve rispettare la decisione presa.

Quando viene deciso un risultato, deve essere lo stesso per tutti. (Fanno tutti
la stessa cosa).

\begin{definition}(Preferenza aggergata)

    Abbiamo un insieme di Agenti $A$ e ogni agente ha una preferenza.

    Relazioni di Preferenza: $a \succsim b$. Indica che anche l'agente $i$ ha una
    preferenza.

    \begin{equation}
        \begin{aligned}
            O = \{a,b,c\dots\} \rightarrow Risultati \\
            A = \{1,2,3\dots\} \rightarrow Agenti    \\
            P = \{ \succsim_1, \succsim_2, \succsim_3\dots\} \rightarrow Preferenze
        \end{aligned}
    \end{equation}
\end{definition}

Una \textbf{funzione di scelta sociale} prende in input tutte le preferenze.

\begin{definition}(Funzione di scelta sociale)
    Dato un insieme di agenti $N = \{1,2,3\dots\}$ e un insieme di risultati $O = \{a,b,c\dots\}$.
    Sia $L\_$ l'insieme di \textbf{total order non stretti} su $O$. Una \textbf{funzione
        di scelta sociale} e' $C: L\_^n \rightarrow O$.
\end{definition}

\begin{definition}(Funione di welfare sociale)
    Dato un insieme di agenti $N = \{1,2,3\dots\}$ e un insieme di risultati $O = \{a,b,c\dots\}$.
    Sia $L\_$ l'insieme di \textbf{total order non stretti} su $O$. Una \textbf{funzione
        di welfare sociale} e' $W: L\_^n \rightarrow L\_$.
\end{definition}

Qual e' la differenza tra una funzione di scelta sociale e una funzione di
welfare sociale?

\begin{itemize}
    \item Una funzione di scelta sociale prende in input tutte le preferenze e
          restituisce un risultato.
    \item Una funzione di welfare sociale prende in input tutte le preferenze e
          restituisce un ordinamento.
\end{itemize}

\subsection{Sistema di voto senza ranking}

Praticamente ci si interessa ad un solo candidato.

Immaginiamo di avere $4$ agenti (Per la pluralita' che vedremo ora).
\begin{itemize}
    \item \textbf{Pluralita'}: Il candidato con piu' voti vince.
          \begin{table}[H]
              \centering
              \begin{tabular}{|c|c|c|c|}
                  \hline
                  \textbf{Agente} & \textbf{a} & \textbf{b} & \textbf{c} \\
                  \hline
                  \textbf{Voto}   & $a$        & $b$        & $a$        \\
                  \hline
              \end{tabular}
              \caption{Esempio di pluralita'}
              \label{tab:my_label}
          \end{table}

    \item \textbf{Voto cumulativo}: Ad esempio distribuire piu voti tra i candidati. Si puo' votare un agente piu volte.
          \begin{table}

          \end{table}
    \item \textbf{Voto ad approvazione}: Accetta quanti risultati "preferisci".
\end{itemize}

\subsection{Sistema di voto con ranking}

In questo caso vogliamo un \textbf{ranking} e non un solo vincitore.

Diamo qualche nozione come prima:

\begin{itemize}
    \item \textbf{Pluralita' con eliminazione}: Si eliminano i candidati con meno voti e si ripete il processo fino ad ottenere un solo candidato.
    \item \textbf{Borda}:
          \begin{itemize}
              \item Si associa ad ogni risultato un numero.
              \item Il risultato preferito prende un punteggio di $n-1$. Il prossimo dopo di lui
                    prende $n-2$, fino ad arrivare all'$n-esimo$ che prendera' $n-n$ cioe $0$.
              \item Si sommano i numeri di ogni risultato e si prende il risultato con il punteggio
                    piu' alto.
          \end{itemize}
    \item \textbf{Eliminazione a coppie}: Si prendono in anticipo le coppie di risultati da confrontare. Per ogni coppia, si elimina quello con meno punti. Si continua cosi fino ad arrivare alla fine.
\end{itemize}

\begin{domanda}
    (Quale tra questi schemi e' il migliore?)

    La domanda mi devasta... No, questo e' un problema di informatica. Bisogna
    ragionare sulle seguenti intuizioni.

    Se c'e' un candidato che e' il preferito da tutti e che batte qualsiasi altro
    candidato in una sfida a coppie, allora lui si \textbf{merita} di vincere.

    C'e' da dire, pero', che questo schema non e' sempre applicabile. Alcune volte
    potremmo trovare unc ciclo dove $A$ batte $B$. $B$ batte $C$ e poi $C$ batte
    $A$ in una sfida a coppie. Una bella gatta da pelare.
\end{domanda}

\subsection{Fun Game}

\begin{esempio}
    (Fun Game - TRIP)

    Immaginiamo ci sia la possibilita' di organizzare una settimana alla fine delle
    lezioni in una delle seguenti destinazioni.

    \begin{itemize}
        \item \textbf{O} - Orlando
        \item \textbf{P} - Parigi
        \item \textbf{T} - Tokyo
        \item \textbf{B} - Beijing
    \end{itemize}

    Dobbiamo costruire il nostro ordine di preferenza. Cosa usiamo?

    Votiamo usando i seguenti schemi:
    \begin{itemize}
        \item Pluralita
        \item Pluralita' con eliminazione
        \item Borda
        \item Eliminazione a coppie
    \end{itemize}
\end{esempio}

\begin{esempio}
    (Esempio Condorcet)

    \begin{equation}
        \begin{aligned}
            499 \ Agents & : A \succ B \succ C \\
            3 \ Agents   & : B \succ C \succ A \\
            498 \ Agents & : C \succ B \succ A \\
        \end{aligned}
    \end{equation}

    Chi e' il vincitore secondo il metodo di Condorcet? Ovviamente $B$ perche' non
    ci sono casi in cui $B$ e' il meno preferito.

    Chi vincerebbe, invece, sotto il caso di Pluralita'? $A$ perche' e' il piu'
    preferito. \textit{Spiegazione:}Contando i voti per il candidato preferito da
    ogni agente, otteniamo:

    A riceve 499 voti. B riceve 3 voti. C riceve 498 voti. Quindi, secondo il
    metodo di Pluralità, A vince perché ha ricevuto il maggior numero di voti.

    \textbf{Attenzione pero'}. Se secondo il metodo condorcet si sceglie il piu intuitivo,
    come fa ad essere $A$ in questo caso il vincitore? Siamo osservando un
    \textbf{paradosso.}

    Chi vincerebbe, invece, sotto Pluralita' con eliminazione? 
    Vincerebbe $C$. Si elimina il meno preferito, quindi $B$. Dopodiche' i 
    voti per C sono quelli del $498$ agenti + 3 per cui vince C su A.
\end{esempio}

\subsection{Sensitivita' al candidato perdente}

Immaginiamo una situazione del genere:

\begin{equation}
    \begin{aligned}
        35 \ Agents & : A \succ C \succ B \\
        33 \ Agents & : B \succ A \succ C \\
        32 \ Agents & : C \succ B \succ A \\
    \end{aligned}
\end{equation}

Quale candidato vincerebbe sotto \textbf{pluralita'}? $A$.

Quale candidato vincerebbe sotto \textbf{Borda}?

Come ragioniamo? Assegnamo i punteggi e sommiamo.

$A \rightarrow 2; B \rightarrow 2; C \rightarrow 2$.

$C \rightarrow 1; B \rightarrow 1; A \rightarrow 1$.

$B \rightarrow 0; A \rightarrow 0; C \rightarrow 0$.

\begin{equation}
    \begin{aligned}
        A &= 2 \cdot 35 + 1 \cdot 33 + 0 \cdot 32 = 103 \\
        B &= 2 \cdot 33 + 1 \cdot 32 + 0 \cdot 35 = 98  \\
        C &= 2 \cdot 32 + 1 \cdot 35 + 0 \cdot 33 = 99  \\
    \end{aligned}
\end{equation}

Chiaramente, il vincitore e' $A$.

Cosa succederebbe se eliminassimo $C$ dal gioco? Cosa succede sia per una che per l'altra modalita' di voto? \textbf{Vincerebbe} $B$.

Se invece giocassimo con \textbf{eliminazione a coppie} con ordine $A,B,C?$ 
Vincerebbe $C$. Perche'? Perche' $C$ batte $A$ e $B$.

E con un ordine $A,C,B$? Vincerebbe $B$

E con un ordine di $B,C,A$? Vincerebbe $A$

\textbf{Costruiamo un altro problema}

\begin{equation}
    \begin{aligned}
        1 \ Agent & : B \succ D \succ C \succ A \\
        1 \ Agent & : A \succ B \succ D \succ C \\
        1 \ Agent & : C \succ A \succ B \succ D \\
    \end{aligned}
\end{equation}

Usando \textbf{eliminazione a coppie} con ordine $A,B,C,D$? 
Vediamo se ho capito bene.

\begin{itemize}
    \item $A$ vs $B$ : Vince $A$
    \item $A$ vs $C$ : Vince $C$
    \item $D$ vs $C$ : Vince $D$
\end{itemize}

Il vincitore e' $D$.

\textbf{Problema}: Tutti gli agenti preferiscono $B$ a $D$. Il candidato e' \textbf{Pareto dominato!}

Ogni metodo ha dei problemi. Sta a te capire quale scegliere.

Definiamo delle notazioni e proprietà per risolvere i problemi come quello precedente.

\begin{itemize}
    \item $N$: Insieme di Agenti
    \item $O$: Insieme finito di risultati con $|O| \geq 3$
    \item $L$: L'insieme di tutte le preferenze di ordinamento stretto su $O$.
    \item $[\succ]$: Un elemento dell'insieme $L^n$ (Un ordinamento di preferenza per ogni agente. L'input della nostra funzione di welfare sociale)
    \item $\succ_w$: L'ordine di preferenza selezionato dalla funzione di benessere sociale $W$
    \begin{itemize}
        \item Quando l'input a $W$ è ambiguo lo scriviano nel subscript. L'ordine sociale selezionato da $W$
        è scritto come $\succ_{w}([\succ'])$.
    \end{itemize}
\end{itemize}

\begin{definition}
    (Efficienza di Pareto)

    $W$ è \textbf{Pareto efficiente} se per qualsiasi $o_1, o_2 \in O$,
    $\forall i o_1 \succ_i o_2$ implica che $o_1 \succ_w o_2$.

    Cioè, se tutti gli agenti preferiscono $o_1$ a $o_2$, allora $W$ deve scegliere $o_1$ a $o_2$.
\end{definition}

\begin{definition}
    (Non dittatorialità)

    $W$ non ha un \textbf{dittatore} se $\lnot \exists i forall o_1, o_2 (o_1 \succ_i o_2 \implies o_1 \succ_w o_2)$.

    Cioé, non esiste un singolo agente le quali preferenze determinano 
    sempre l'ordine sociale. Altrimenti, diciamo che $W$ è \textbf{dittatoriale}.
\end{definition}

\begin{definition}
    (Arrow Theorem)

    Una qualsiasi funzione di ordine sociale $W$ che è Pareto Efficiente e indipendente 
    da relative irrilevanti è dittatoriale.

\end{definition}

La dimostrazione mi rifiuto di farla.


\newpage