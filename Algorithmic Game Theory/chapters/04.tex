\section{Giochi Competitivi o Non Cooperativi}

I giochi competitivi o non cooperativi sono i più studiati. Si assume che un
numero di agenti agisca strategicamente e sia interessato solo al proprio
guadagno, senza collaborare. Cosa succede quando ogni agente ha un obiettivo e
alcuni interessi, e forse alcuni di essi sono in contrasto?

Abbiamo un insieme di agenti che chiamiamo $N$ (per esempio $N =
    \{1,2,\ldots\}$). Ognuno di essi ha alcune azioni possibili tra cui scegliere,
quindi abbiamo un insieme di azioni $A_i$ che è un sottoinsieme di tutte le
possibili azioni $A_i \subset A$ (per esempio $A_i = \{p, q, \ldots\}$). Quando
scegliamo un'azione, otteniamo un valore per questa azione, e c'è un premio
associato che è un numero reale. Questo premio dipende non solo dalle nostre
azioni ma anche dalle azioni degli altri giocatori, quindi il valore dipende da
noi e dagli altri agenti.

Ciascun agente è associato a un'utilità $u_i$ che dipende da tutte le
strategie:
\[u_i : A_1 \times A_2 \times A_3 \times \ldots \times A_n \to \mathbb{R}\]

Il nostro obiettivo è massimizzare la nostra utilità, ma non possiamo
ottimizzare completamente poiché non possiamo controllare le variabili degli
altri giocatori. Dobbiamo anche considerare cosa giocheranno gli altri
giocatori.

Consideriamo un gioco con due giocatori, Bob e John. Le azioni disponibili sono
$\{HOME, OUT\}$. Cosa succede quando eseguono queste azioni? Di solito,
mettiamo l'utilità di entrambi gli agenti in una matrice:

\[
    \begin{array}{ccc}
                    & \text{HOME} & \text{OUT} \\
        \text{HOME} & (1, 2)      & (1, 0)     \\
        \text{OUT}  & (2, 0)      & (-1, 3)    \\
    \end{array}
\]

In questo esempio, i numeri rappresentano le preferenze mappate su numeri
reali.

Cosa significa questo? Sono una mappatura delle preferenze in numeri reali.
Quale sarebbe l'esito? Di solito non è facile capire cosa accadrà.

\[
    \begin{array}{ccc}
                    & \text{HOME} & \text{OUT} \\
        \text{HOME} & (2, 2)      & (1, 1)     \\
        \text{OUT}  & (1, 0)      & (0, 0)     \\
    \end{array}
\]

In questo esempio, l'equilibrio di Nash è \{HOME, HOME\}, poiché questo esito è
fortemente preferito tra gli scenari. Ma cosa succede in questo caso?

\[
    \begin{array}{ccc}
                    & \text{HOME} & \text{OUT} \\
        \text{HOME} & (2, 2)      & (1, 3)     \\
        \text{OUT}  & (1, 0)      & (0, 0)     \\
    \end{array}
\]

(1,3) è un equilibrio di Nash. (2,0) è un equilibrio di Nash.

\subsection{Equilibrio di Nash in Giochi Puri}

Una strategia $s_i$ per un agente è un'azione, in cui scegliamo una delle
azioni disponibili. Un profilo di strategia congiunta o $\sigma$ è una
strategia per ciascun agente del gioco. $\sigma$ è un equilibrio di Nash se,
per ciascun agente possibile, l'utilità dell'agente ($u(\sigma)$) ottenuta per
una particolare strategia che decide di giocare dovrebbe essere almeno uguale o
maggiore dell'utilità che potrebbe ottenere se gli altri giocassero la stessa
strategia, ma lui prova una nuova strategia. In altre parole, giocando una
strategia diversa mentre gli altri mantengono le stesse strategie, dovrebbe
ottenere lo stesso valore o un valore maggiore. $\sigma = (\sigma_i$ che è
$s_i$, $\sigma_{-i}$ è la strategia degli altri).

Questo tipo di equilibrio e impostazione è chiamato "puro". Un esempio può
aiutare a capire il concetto:

\[
    \begin{array}{ccc}
                    & \text{HOME} & \text{OUT} \\
        \text{HOME} & (1, 2)      & (1, 0)     \\
        \text{OUT}  & (2, 0)      & (-1, 3)    \\
    \end{array}
\]

In questo caso, non esiste un equilibrio di Nash, nessuno è davvero felice, e
c'è una situazione ciclica. Quando ci concentriamo sull'equilibrio di Nash,
possiamo avere uno, più di uno o nessuno.

Per calcolare un equilibrio di Nash, è necessario iterare tra le strategie
degli agenti e tra le strategie di ciascun agente. Il limite è dominato dalla
dimensione del prodotto cartesiano: $|A_1 \times A_2 \ldots \times A_n| \times
    |N| \times \max|A_i| \geq 2^n$. Ciò rende difficile rappresentare matrici
quando il numero di agenti è grande, poiché la dimensione della
rappresentazione è esponenziale rispetto al numero di agenti.

\subsection{Equilibrio di Nash in Giochi Grafici}

Ma cosa succede con l'equilibrio di Nash in giochi non puri? Supponiamo di
avere un grande gioco di popolazione, interagirai davvero con tutti i
partecipanti al gioco? Invece, interagirai con i tuoi vicini (amici e nemici),
quindi creerai un grafo con nodi che corrispondono esattamente agli agenti.
L'utilità dipende dalle tue azioni e dalle azioni dei tuoi vicini. Questo è
noto come "giochi grafici". I tuoi vicini sono un sottoinsieme di $N$ e sono i
nodi adiacenti. La funzione di utilità è $u_i : A_i \times \prod_{j \in
        \text{neigh}_i} A_j \to \mathbb{R}$. Se hai al massimo $k$ agenti intorno a te,
allora sappiamo che la dimensione (la matrice, l'utilità) è di circa $O(2^k)$,
ma poiché $k$ è limitato, questa rappresentazione è ragionevole.

Tuttavia, è importante notare che anche se non sei direttamente connesso, gli
altri influenzano comunque i tuoi vicini, quindi alla fine è necessario
considerare tutti i giochi e tutti gli agenti.

\subsection{Complessità Computazionale}

Consideriamo il problema: esiste un equilibrio di Nash in un gioco grafico
(IMPOSTAZIONE PURA)? Siamo in NP? - Dobbiamo indovinare un profilo di
strategia, il che richiede tempo polinomiale, quindi la dimensione di $\sigma$
è polinomiale rispetto alle dimensioni dei giochi. - Verifica se il profilo di
strategia è un equilibrio di Nash: dobbiamo iterare su tutti gli agenti
possibili e su tutte le possibili azioni $|N| \times |A_i|$, il che è anch'esso
polinomiale.

Quindi sappiamo che questo problema è in NP. Ma è anche NP-hard?

Supponiamo di avere due agenti e tre azioni:

\[
    \begin{array}{cccc}
          & R     & G     & B     \\
        R & (0,0) & (1,1) & (1,1) \\
        G & (1,1) & (0,0) & (1,1) \\
        B & (1,1) & (1,1) & (0,0) \\
    \end{array}
\]

\subsection{Esempio di Equilibrio di Nash}

In questo esempio, cerchiamo di capire come ottenere un equilibrio di Nash in
un gioco usando un esempio di colorazione di grafo. Supponiamo che vogliamo che
due agenti, a e b, siano felici solo se scelgono colori diversi. Abbiamo tre
colori: Rosso (R), Verde (G) e Blu (B).

\[
    \begin{array}{cccccccc}
          &   &       & a     &       &            &       & \\
          &   & R     &       & G     &            & B     & \\
        a & R & (0,0) & (1,1) & (1,1) & (1/2, 1/2)           \\
          & G & (1,1) & (0,0) & (1,1)                        \\
          & B & (1,1) & (1,1) & (0,0)                        \\
          & U &       &       &       &            & (2,2)   \\
          &   & R     &       & G     &            & B     & \\
          &   &       & b     &       &            &       & \\
    \end{array}
\]

In questo modo, possiamo sempre far sì che gli agenti siano felici quando
scelgono colori diversi. Supponiamo che il grafo sia 3-colorabile, il che
significa che è possibile colorare i nodi in modo che nessun nodo adiacente
abbia lo stesso colore. In questo caso, esiste un equilibrio di Nash.

Supponiamo ora che il grafo non sia 3-colorabile, il che significa che ci
saranno sempre due nodi che sceglieranno lo stesso colore, ma per uno di loro
non sarà conveniente cambiarlo. Questo "ingranaggio" potrebbe non funzionare,
ma potrebbe sempre esistere un equilibrio di Nash.

Inoltre, aggiungiamo un'ulteriore variabile, $U$, che rappresenta la situazione
in cui non è possibile ottenere un colore diverso e quindi giochiamo $U$. In
questa situazione, se un vicino sta giocando $U$, vogliamo anche noi giocare
$U$. Ma in questo modo potremmo finire in un equilibrio di Nash in cui tutti
giocano $U$, il che non vogliamo. Per affrontare questa situazione, aggiungiamo
    un giocatore fittizio, detto anche "dummy player", che non è felice a giocare
$U$, ma si preoccupa solo se qualcun altro sta giocando $U$. Abbiamo quindi due
    giocatori, $\alpha$ e $\beta$, che giocheranno come segue:

\[
    \begin{array}{ccc}
               & \alpha & \beta  \\
        \alpha & (1,-1) & (-1,1) \\
        \beta  & (-1,1) & (1,-1) \\
    \end{array}
\]
In questo modo, otteniamo un'utilità che non scatena una situazione di
soddisfacibilità. Pertanto, assumendo che il grafo non sia 3-colorabile,
potrebbe non esistere un equilibrio di Nash.

\newpage