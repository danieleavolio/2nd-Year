\documentclass{article}

% Packages for code, figures, and automata
\usepackage{listings} % For code listings
\usepackage{graphicx} % For including figures
\usepackage{tikz}     % For drawing automata
\usepackage{multirow}
\usepackage{array}
\usepackage{amssymb} % Pacchetto per simboli matematici
\usepackage{float}

%colorize lstlisting with language
\usepackage{xcolor}

\usepackage{titlesec}

%import all important packages 

% Configurazione degli stili per tutti i linguaggi
\lstset{
    basicstyle=\ttfamily,
    keywordstyle=\color{blue},
    commentstyle=\color{green},
    stringstyle=\color{red},
    % Altre opzioni
    breaklines=true, showstringspaces=false,
    emph={label},
    emphstyle={\color{custompurple}},
    escapeinside={(*}{*)}
    }

% Stile globale per tutti i linguaggi
\lstdefinestyle{mystyle}{
    backgroundcolor=\color{white},
    commentstyle=\color{green},
    keywordstyle=\color{blue},
    numberstyle=\tiny\color{gray},
    stringstyle=\color{red},
    basicstyle=\ttfamily\footnotesize,
    breakatwhitespace=false,
    breaklines=true,
    captionpos=b,
    keepspaces=true,
    numbers=left,
    numbersep=5pt,
    showspaces=false,
    showstringspaces=false,
    showtabs=false,
    tabsize=2
}

% Impostazioni per tutti i linguaggi
\lstset{style=mystyle}
% Definizione di simboli per subsection e subsubsection
\newcommand{\subsecsymbol}{\textcolor{custompurple}{\rule[0pt]{10pt}{10pt}\hspace{10pt}}}
\newcommand{\subsubsecsymbol}{\textcolor{custompurple}{\textbf{$\blacklozenge$}\hspace{4pt}}}

\titleformat{\section}[block]
  {\Huge\bfseries}
  {\llap{\textcolor{custompurple}{\rule[-4pt]{10pt}{18pt}\hspace{10pt}}\thesection\hskip 12pt}}
  {0pt}
  {}
% Definizione di uno stile per \subsection
\titleformat{\subsection}[block]
  {\Large\bfseries\color{black}}
  {\llap{\subsecsymbol}\thesubsection\hskip 12pt}
  {0pt}
  {}

% Definizione di uno stile per \subsubsection
\titleformat{\subsubsection}[block]
  {\large\bfseries\color{black}}
  {\llap{\subsubsecsymbol}\thesubsubsection\hskip 12pt}
  {0pt}
  {}



%make link clickable
\usepackage{hyperref}
\usepackage{pgfplots}

%use asmath
\usepackage{amsmath}

\usepackage{fancyhdr}
\pagestyle{fancy}
\fancyhf{}
\fancyhead[R]{\nouppercase{\rightmark}}
\fancyfoot[C]{\thepage}

\usepackage{l   istings}
\usepackage{tabularx}

%color link orange
\hypersetup{
    colorlinks=true,
    linkcolor=custompurple,
    filecolor=magenta,
    urlcolor=cyan,
}

\definecolor{custompurple}{HTML}{8b3fff}


% Titolo e autore del documento
\title{Algorithmic Game Theory}
\author{Daniele Avolio}
\date {A.A. 2023/2024}


\begin{document}

\maketitle
\newpage

\tableofcontents
\newpage

% Includi qui i tuoi capitoli o sezioni
\section{Introduzione}
Appunti 
\newpage

\section{Deep Learning 101}

In questo corso affronteremo diverse tematiche, il che può sembrare assurdo se ci si pensa.

\begin{itemize}
  \item Classificazione (binaria, cioè sì o no)
  \item Multi-class classification (non più binaria)
  \item Regressione (il guessing viene fatto su un valore numerico)
  \item Gestione di immagini e riconoscimento
  \item Serie numeriche (predizioni di mercato e trend)
  \item Classificazione di testi
\end{itemize}

\subsection{Architetture e strumenti nel deep learning}

\begin{itemize}
  \item Autoencoder
    \begin{itemize}
      \item Tutti i possibili tipi
      \item Qui si fa anche \textbf{Clustering} e \textbf{Anomaly detection}
    \end{itemize}
  \item Architetture generative
    \begin{itemize}
      \item Tutti i possibili tipi
    \end{itemize}
  \item XAI: Explainable AI
\end{itemize}

\subsection{Libri utili}
\begin{itemize}
  \item "Deep Learning in Python"
  \item "Tensorflow tutorial"
\end{itemize}

\subsection{Strumenti che useremo}
\begin{itemize}
  \item Tensorflow
    \begin{itemize}
      \item High-level più di altri
    \end{itemize}
  \item Keras
    \begin{itemize}
      \item High-level API basato su Tensorflow
      \item Ci saranno cose che non possiamo fare con Keras perché è troppo ad alto livello
    \end{itemize}
\end{itemize}

\subsection{Schema generale di un problema di deep learning}

Abbiamo delle coppie $(x_0,y_0), (x_1,y_1), (x_2,y_2), \ldots, (x_n, y_n)$ dove $x_i$ è un vettore di features e $y_i$ è un valore numerico (regressione) o una classe (classificazione).

$$y_i = f(x_i)$$

Non conosciamo la funzione $f$, quindi dobbiamo impararla.

$$y = \alpha x + \beta$$

Con una rete neurale puoi approssimare praticamente qualsiasi funzione.

\textit{Una rete neurale permette di collegare un input di dati a una funzione di output.}

Abbiamo diversi tipi di reti neurali a seconda del tipo di problema che vogliamo risolvere. È importante essere in grado di selezionare l'architettura giusta per risolvere il problema.

Definiamo alcuni concetti che useremo:
\begin{itemize}
  \item $N$: rete neurale
  \item $w$: valori dei pesi della rete neurale
  \item $f$: funzione di output della rete neurale
\end{itemize}

$$ f \in N(w)$$

\subsection{Perché si usa il termine "Tensore"?}

Un \textit{tensore} non è altro che una matrice.
\begin{itemize}
  \item 0D tensor: scalar
  \item 1D tensor: vector
  \item 2D tensor: matrix
  \item 3D tensor: tensor
\end{itemize}

\subsection{AI vs DL}

\begin{itemize}
  \item AI: è un ampio insieme di tecniche per risolvere problemi che richiedono "intelligenza".
    \begin{itemize}
      \item Esempio: Stockfish, un programma che gioca a scacchi.
    \end{itemize}
  \item DL: È un sottoinsieme di AI che si concentra sull'\textit{astrazione}.
    \begin{itemize}
      \item L'astrazione consiste nel fornire una funzione che traduce dati di input in dati di output senza conoscere la funzione stessa.
      \item È un approccio induttivo: fornisci un input e ti aspetti un output, senza conoscere la funzione che li collega.
      \item Questo è completamente diverso dall'AI basata sulla logica.
    \end{itemize}
\end{itemize}


\newpage
\section*{Appunti di Laboratorio}
\section{Introduzione alla teoria dell'utilità e decision making}
\subsection{Concetti di base}
\subsubsection{ALTERNATIVE}
Parliamo di \textit{agenti} che devono scegliere un'\textit{alternativa} da
un'insieme $\mathcal{X}$ di alternative. Questo insieme di alternative ha degli
elementi che possono essere \textbf{esaustivi} o \textbf{mutualmente
    esclusivi}.

\textbf{Esepmio}: $\mathcal{x}$ \{
\begin{itemize}
    \item DL = Deep Learning
    \item AGT = Algorithmic Game Theory
    \item DLAGT = Deep Learning Algorithmic Game Theory
    \item N = None
\end{itemize}
\}

\subsubsection{PREFERENZE}

Con il termine \textbf{preferenze} identifichiamo una relazione $\succcurlyeq$
su $\mathcal{X}$, che è un sottoinsieme di $\mathcal{X} \times \mathcal{X}$. Le
preferenze possono essere:
\begin{itemize}
    \item \textbf{complete} se $\forall x,y \in \mathcal{X}$ vale $x \succcurlyeq y$ oppure $y \succcurlyeq x$
    \item \textbf{transitive} se $\forall x,y,z \in \mathcal{X}$ vale $x \succcurlyeq y$ e $y \succcurlyeq z$ allora $x \succcurlyeq z$
\end{itemize}

\subsubsection{Relazione di Preferenza}

Una preferenze è una \textbf{relazione di preferenza} se è sia
\textbf{completa} che \textbf{transitiva}.

Si chiama preferenza \textbf{stretta} se $x \succ y \iff x \succcurlyeq y$ e $x
    \nsucceq y$.

Si chiama \textbf{indifferenza} se $x \sim y \iff x \succcurlyeq y$ e $x
    \preccurlyeq y$.

\subsubsection{Rappresentare le preferenze come \textbf{utilità}}

Una relazione di preferenza puà essere tradotta in una funzione di utilità del
tipo $u: \mathcal{X} \rightarrow \mathbb{R}$. Si può fare in questo modo:

\begin{equation}
    x \succcurlyeq y \iff u(x) \geq u(y) \quad \forall x,y \in \mathcal{X}
\end{equation}

\textit{Esempio:} Se un agente dovesse trovare \textit{x} almeno buono quanto \textit{y}, allora la funzione di utilità $u(x)$ deve essere almeno
alta quanto $u(y)$. Cioè l'agente è come se stesse \textbf{massimizzando} il valore di $u(var)$.

\textbf{Rappresentazione ordinale}
\textbf{Teorema 1}: Sia $\mathcal{X}$ un insieme finito di alternative e sia $\succcurlyeq$ una relazione di preferenza su $\mathcal{X}$.
Allora una preferenza può essere rappresentata come una funzione di utilità $u: \mathcal{X} \rightarrow \mathbb{R}$ se e solo se è \textbf{completa} e \textbf{transitiva}.
In più, se $f:\mathcal{R} \rightarrow \mathcal{R}$ è una funzione monotona crescente, allora $f \circ u$ rappresenta la stessa preferenza di $u \succcurlyeq$

\textbf{Nota}: dall'ultimo statement, l'ordine ha rilevanza.

Per essere valido ci sono 2 condizioni necessarie:
\begin{itemize}
    \item Transitività: Cioè, dato $\mathcal{X} = \{a,b,c\}$, suponiamo che $a \succ b
              \succ c \succ a \implies u(a) > u(b) > u(c) > u(a)$. Questo sarebbe
          \textbf{assurdo}.
    \item Completezza: Se abbiamo preferenze incomplete, allora al massimo possiamo
          costruire un ordine per un sottoinsieme di $\mathcal{X}$.
\end{itemize}

\textbf{Dimostrazione}

La transitività e la completezza sono necessarie e sufficienti. Supponiamo di
avere l'insieme $X = \{X_1, \ldots, X_n\}$. Possiamo suddividere gli elementi
di $X$ in $k$ classi di indifferenza $C_1, \ldots, C_k$ tali che $C_1 \succ C_2
    \succ \ldots \succ C_k$. In questo modo, possiamo definire la funzione di
utilità $u$ in modo che:

\[
    u(x) = k \quad \forall x \in C_1,
\]
\[
    u(x) = k-1 \quad \forall x \in C_2,
\]
\[
    \ldots
\]
\[
    u(x) = 1 \quad \forall x \in C_k.
\]

In questo contesto, $\succ$ rappresenta la relazione di preferenza.

\subsection{Le lotterie}

Una lotteria è una tupla $\mathcal{L} = (p_1,x_1;p_2,x_2 \ldots, p_n,x_n)$.

\begin{itemize}
    \item Con prezzo monetario $x_1,x_2, \dots, x_n \in X \subseteq R$.
    \item Distribuzione di probabilità $(p_1, p_2, \dots, p_n)$.
\end{itemize}

Quindi con $\mathcal{L}$ viene definito l'insieme delle lotterie semplici.

\end{document}
