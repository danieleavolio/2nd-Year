\section{Domande: Big Data Introduction}
\begin{domanda}
    Cosa sono i Big Data?
\end{domanda}
I Big Data sono un insieme di dati che viene differenziato dai comuni dati
dal fatto che per essere gestiti non sono piu' abbastanza le vecchie
tecnologie e metodi di gestione dei dati. I fattori che influenzano sono la velocita con cui
vengono generati, il volume di dati che viene genegrato e la varieta' di dati
che ci sono all'interno del dataset.

\begin{domanda}
    Quali sono le 5 V dei big data?
\end{domanda}

Le 5 V dei big data fanno riferimento alle caratteristiche di essi, che sono:
\begin{itemize}
    \item \textbf{Volume}: La quantita' di dati enorme che vanno gestiti. La scala che va fino ai
          Zettabyte.
    \item \textbf{Varieta}': I vari tipi di danni che sono all'interno del database da analizzare e gestire. Possono essere immagini, audio, video, testo, ecc\dots
    \item \textbf{Velocita}': La velocita' con cui i dati vengono generati e devono essere analizzati.
    \item \textbf{Veracita}': L'incertezza e l'imprecisione dei dati che vengono generat e che quindi devono essere analizzati per capirne la \textit{qualita'}.
    \item \textbf{Valore}: Il valore che i dati hanno per l'azienda che li possiede. Ad esempio vantaggi di business e analisi possibili da condurre.
\end{itemize}

\begin{domanda}
    Qual e' stata la prima rivoluzione nei database?
\end{domanda}

Boh era quella iniziale dove praticamente si usava un modello gerarchico e veniva utilizzato Cobol.
Diciamo che per gestirne uno sarebbe servito un esperto di database.

\begin{domanda}
    Qual e' stata la seconda rivoluzione nei database?
\end{domanda}

L'introduzione del \textbf{modello relazionale} che porta i dati da essere tutti in unici tabelle a diverse con una struttura ben specifica.

Qui nasce anche il concetto di \textbf{transzione} che deve essere ACID:
\begin{itemize}
    \item A: Atomica
    \item C: Consistente
    \item I: Isolata
    \item D: Duratura
\end{itemize}

\begin{domanda}
    Qual e' stata la terza rivoluzione nei database?
\end{domanda}

Quando sono stati inventati i social network, grandissime moli di dati sono cominciate ad essere generate.
Le strutture di un tempo non erano ancora abbastanza solide da poter gestire delle moli di dati tali. Per questo motivo \textbf{nuove tecniche} sono state inventate.
Quindi nascono anche i database non relazionali.
Ci sono state varie invenzioni come \textbf{lo sharding}. Sono stati creati altri tipi di database come quelli \textbf{(chiave, valore)}.
Principalmente, pero', fu Google con il \textbf{distributed file system} a dare il via a questa rivoluzione.

Un concorrente a google fu Yahoo con HDFS (Hadoop Distributed File System).

\newpage