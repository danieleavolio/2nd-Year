\documentclass{article}

% Packages for code, figures, and automata
\usepackage{listings} % For code listings
\usepackage{graphicx} % For including figures
\usepackage{tikz}     % For drawing automata
\usepackage{multirow}
\usepackage{array}
\usepackage{amssymb} % Pacchetto per simboli matematici
\usepackage{float}

\usepackage{pgfplots}
\usepackage{amsmath}

%colorize lstlisting with language
\usepackage{xcolor}

\usepackage{titlesec}

%package for theorem
\usepackage{amsthm}
\newtheorem*{theorem}{theorem}

%import all important packages 

% Configurazione degli stili per tutti i linguaggi
\lstset{
    basicstyle=\ttfamily,
    keywordstyle=\color{blue},
    commentstyle=\color{purple},
    stringstyle=\color{red},
    % Altre opzioni
    breaklines=true, showstringspaces=false,
    emph={label},
    emphstyle={\color{custompurple}},
    escapeinside={(*}{*)}
    }

% Stile globale per tutti i linguaggi
\lstdefinestyle{mystyle}{
    backgroundcolor=\color{white},
    commentstyle=\color{purple},
    keywordstyle=\color{blue},
    numberstyle=\tiny\color{gray},
    stringstyle=\color{red},
    basicstyle=\ttfamily\footnotesize,
    breakatwhitespace=false,
    breaklines=true,
    captionpos=b,
    keepspaces=true,
    numbers=left,
    numbersep=5pt,
    showspaces=false,
    showstringspaces=false,
    showtabs=false,
    tabsize=2
}

% Impostazioni per tutti i linguaggi
\lstset{style=mystyle}
% Definizione di simboli per subsection e subsubsection
\newcommand{\subsecsymbol}{\textcolor{custompurple}{\rule[0pt]{10pt}{10pt}\hspace{10pt}}}
\newcommand{\subsubsecsymbol}{\textcolor{custompurple}{\textbf{$\blacklozenge$}\hspace{4pt}}}

\titleformat{\section}[block]
  {\Huge\bfseries}
  {\llap{\textcolor{custompurple}{\rule[-4pt]{10pt}{18pt}\hspace{10pt}}\thesection\hskip 12pt}}
  {0pt}
  {}
% Definizione di uno stile per \subsection
\titleformat{\subsection}[block]
  {\Large\bfseries\color{black}}
  {\llap{\subsecsymbol}\thesubsection\hskip 12pt}
  {0pt}
  {}

% Definizione di uno stile per \subsubsection
\titleformat{\subsubsection}[block]
  {\large\bfseries\color{black}}
  {\llap{\subsubsecsymbol}\thesubsubsection\hskip 12pt}
  {0pt}
  {}


%make link clickable
\usepackage{hyperref}
\usepackage{pgfplots}

%use asmath
\usepackage{amsmath}

\usepackage{fancyhdr}
\pagestyle{fancy}
\fancyhf{}
\fancyhead[R]{\nouppercase{\rightmark}}
\fancyfoot[C]{\thepage}

\usepackage{listings}
\usepackage{tabularx}

%color link orange
\hypersetup{
    colorlinks=true,
    linkcolor=custompurple,
    filecolor=magenta,
    urlcolor=cyan,
}

\definecolor{custompurple}{HTML}{8b3fff}

% Define title and author
\begin{titlepage}
    \title{\color{custompurple}\Huge Big Data and Reasoning}
    \author{\textbf{Daniele Avolio}}
    \date{Anno 2023/24}
\end{titlepage}

\begin{document}

\maketitle
\newpage

\tableofcontents

\newpage
\section{Teoria}
In questi appunti ci saranno semplicemente gli appunti del corso di Big Data and Reasoning, tenuto dal professor Francesco Ricca presso l'Università della Calabria.

Non aspettatevi 100\% di correttezza, sono appunti presi a lezione e quindi potrebbero contenere errori!


\newpage
\section{}
\newpage
\section{Lezioni di Laboratorio}

%fai ricominciare il counter sezioni da 1
\newpage

\section*{Domande dell'esame possibili}
In questa sezione ci sono domande che mi vengono fuori quando 
leggo i PDF e che quindi mi aspetto che possano essere 
inserite all'interno delle prove intermedie \/ orale.

Buona fortuna a tutti!

\newpage
\section{Domande: Big Data Introduction}
\begin{domanda}
    Cosa sono i Big Data?
\end{domanda}
I Big Data sono un insieme di dati che viene differenziato dai comuni dati
dal fatto che per essere gestiti non sono piu' abbastanza le vecchie
tecnologie e metodi di gestione dei dati. I fattori che influenzano sono la velocita con cui
vengono generati, il volume di dati che viene genegrato e la varieta' di dati
che ci sono all'interno del dataset.

\begin{domanda}
    Quali sono le 5 V dei big data?
\end{domanda}

Le 5 V dei big data fanno riferimento alle caratteristiche di essi, che sono:
\begin{itemize}
    \item \textbf{Volume}: La quantita' di dati enorme che vanno gestiti. La scala che va fino ai
          Zettabyte.
    \item \textbf{Varieta}': I vari tipi di danni che sono all'interno del database da analizzare e gestire. Possono essere immagini, audio, video, testo, ecc\dots
    \item \textbf{Velocita}': La velocita' con cui i dati vengono generati e devono essere analizzati.
    \item \textbf{Veracita}': L'incertezza e l'imprecisione dei dati che vengono generat e che quindi devono essere analizzati per capirne la \textit{qualita'}.
    \item \textbf{Valore}: Il valore che i dati hanno per l'azienda che li possiede. Ad esempio vantaggi di business e analisi possibili da condurre.
\end{itemize}

\begin{domanda}
    Qual e' stata la prima rivoluzione nei database?
\end{domanda}

Boh era quella iniziale dove praticamente si usava un modello gerarchico e veniva utilizzato Cobol.
Diciamo che per gestirne uno sarebbe servito un esperto di database.

\begin{domanda}
    Qual e' stata la seconda rivoluzione nei database?
\end{domanda}

L'introduzione del \textbf{modello relazionale} che porta i dati da essere tutti in unici tabelle a diverse con una struttura ben specifica.

Qui nasce anche il concetto di \textbf{transzione} che deve essere ACID:
\begin{itemize}
    \item A: Atomica
    \item C: Consistente
    \item I: Isolata
    \item D: Duratura
\end{itemize}

\begin{domanda}
    Qual e' stata la terza rivoluzione nei database?
\end{domanda}

Quando sono stati inventati i social network, grandissime moli di dati sono cominciate ad essere generate.
Le strutture di un tempo non erano ancora abbastanza solide da poter gestire delle moli di dati tali. Per questo motivo \textbf{nuove tecniche} sono state inventate.
Quindi nascono anche i database non relazionali.
Ci sono state varie invenzioni come \textbf{lo sharding}. Sono stati creati altri tipi di database come quelli \textbf{(chiave, valore)}.
Principalmente, pero', fu Google con il \textbf{distributed file system} a dare il via a questa rivoluzione.

Un concorrente a google fu Yahoo con HDFS (Hadoop Distributed File System).

\newpage
\section{Domande: GoodBye SQL}
\begin{domanda}
    Differenza tra Scale Up e Scale Out
\end{domanda}

Scale Up o scalare verticalmente significa \textit{aggiungere risorse ad una
    macchina sola}. Ad esempio, significa potenziare una macchina tale da poter
gestire piu carico di lavoro. Questo e' un approccio che e' stato utilizzato
per molto tempo e che ha portato a creare macchine sempre piu potenti. Il
problema e' che questo approccio ha un limite fisico, ovvero non si puo' andare
avanti all'infinito. Inoltre, questo approccio e' molto costoso.

Scale Out o scalare orizzontalmente e' una cosa piu economicamente sostenibile,
poiche' si basa sul concetto di \textit{aggiugnere piu macchine che lavorano
    tra loro per gestire il carico di lavoro}. Questo e' notevolmente piu economico
perche' i costi di una macchina sono molto piu bassi rispetto a quelli di una
macchina potenziata.

\begin{domanda}
    Quali sono stati i primi tentativi di scalare orizzontalmente
\end{domanda}

I primi tentativi furono quelli di costruire dei \textbf{Memcached server}, in
cui gli utenti potessero effettuare letture senza andare a toccare i database
principali. Un'altra tecnica era quella della \textbf{replicazione dei dati} su
piu' databases.

\textbf{Quando va bene questo?} Quando le letture sono notevolmente maggiori rispetto alle scritture.
Questo perche' se ci fossero tante scritture, ci sarebbe un bottleneck architetturale, dato che
il database su cui vengono effettuate le scritture continua a rimanere 1 e 1 soltanto.

\begin{domanda}
    Cosa significa Sharding e come viene usato?
\end{domanda}

Sharding significa \textbf{tagliare orizzontalmente un database} per splittarlo
su diverse macchine. Questo comporta un aumento delle performance sopratutto in
scrittura, poiche' aumenta il numero di macchine dove si puo' scrivere, ma
aumenta anche la \textbf{complessita' di gestione del sistema.}

Bisogna anche effettuare il taglio con un criterio, poiche' poi quando i dati
devono essere recuperati si vuole un tempo di accesso comunque accettabile
senza dover effettuare una ricerca su tutti e $n$ i database.

\textbf{Problemoni:}
\begin{itemize}
    \item Complessita': Come gia' detto, gestire un sistema del genere diventa complesso
          e richiede una conoscenza elevata
    \item SQL: Non si puo usare SQL su diverse shard, quindi ci vuole un meccanismo
          dietro le quinte che gestisca il tutto
    \item ACID LOSS: Si perdono le transizioni ACID perche' su piu macchine non esiste,
          poiche' si ritornerebbe ad avere un bottleneck
\end{itemize}

\begin{domanda}
    Spiega il teorema di CAP
\end{domanda}

Il teorema di CAP che sta per \textbf{Consistentcy, Availability e Partition
    Tolerance} dice che \textbf{non si possono avere tutti e 3 i requisiti} in un
sistema distribuito. Provando ad ottenere tutti e 3 contemporaneamente, si
andra' comunque a perdere 1 dei 3.
\begin{itemize}
    \item Consistentcy: Ogni utente deve avere la stessa visualizzazione del contenuto in
          qualsiasi istante
    \item Availability: Ogni richiesta deve essere servita
    \item Partition Tolerance: Il sistema deve funzionare anche se alcune macchine non
          sono disponibili
\end{itemize}

Un esempio pratico: \textit{Immagina di avere un servizio di database
    distribuito per un'applicazione di shopping online. In questo sistema, hai
    utenti che effettuano ordini e aggiornano il proprio carrello degli acquisti in
    tempo reale. Il database è distribuito su più server in diverse località
    geografiche per garantire la ridondanza e la tolleranza ai guasti.}

Coerenza (Consistency): Supponiamo che tu abbia implementato una forte coerenza
dei dati, il che significa che ogni aggiornamento del carrello degli acquisti
di un utente deve essere immediatamente riflesso su tutti i server. Questo
assicura che tutti i server abbiano sempre una vista coerente dei dati.
Tuttavia, quando si verifica una partizione di rete tra due gruppi di server,
il sistema deve scegliere tra attendere la risoluzione della partizione
(rendendo il servizio non disponibile per un certo periodo) o accettare il
rischio di avere carrelli degli acquisti temporaneamente non coerenti tra le
due parti del sistema.

Disponibilità (Availability): Se desideri massimizzare la disponibilità, il
sistema dovrebbe continuare a permettere agli utenti di aggiornare i propri
carrelli degli acquisti, anche se si verifica una partizione di rete tra i
server. Tuttavia, ciò può comportare il rischio di avere dati non coerenti tra
i due lati della partizione durante il periodo di separazione.

Tolleranza alla partizione (Partition Tolerance): Il sistema deve essere in
grado di tollerare le partizioni di rete. Questo significa che il servizio
dovrebbe funzionare in presenza di guasti di rete o partizioni geografiche, ma
potrebbe comportare una temporanea mancanza di coerenza o disponibilità in
situazioni di partizione.

Quindi, in questo esempio, puoi vedere che il teorema CAP si applica. Quando si
verifica una partizione di rete, devi fare una scelta tra coerenza e
disponibilità. Non è possibile garantire contemporaneamente entrambe le
proprietà. Il sistema deve tollerare la partizione ma potrebbe sacrificare la
coerenza o la disponibilità a seconda delle decisioni di progettazione prese.

\begin{domanda}
    Spiega: No Go Zone, Eventual Consistentcy, Strict Consistentcy
\end{domanda}

\begin{itemize}
    \item No Go Zone: $$\text{Consistentcy} \cap \text{Availability} \cap \text{Partition
                  Tolerance}$$
    \item Strict Consistentcy: $$\text{Consistentcy} \cap \text{Partition Tolerance}$$
    \item Eventual Consistentcy: $$\text{Availability} \cap \text{Partition Tolerance}$$
\end{itemize}

\begin{domanda}
    Come funziona Amazon Dynamo?
\end{domanda}

\begin{itemize}
    \item Database non relazionale alternativo.
    \item Accesso basato su chiave primaria.
    \item Dati recuperati da una chiave sono oggetti binari non strutturati.
    \item La maggior parte degli oggetti è piccola, sotto 1 MB.
    \item Dynamo permette di sacrificare la coerenza per garantire disponibilità e
          tolleranza alle partizioni.
\end{itemize}

\textbf{Caratteristiche architetturali chiave}
\begin{itemize}
    \item Consistent hashing: usa l'hash della chiave primaria per distribuire i dati tra
          i nodi del cluster in modo efficiente.
    \item Coerenza regolabile: l'applicazione può bilanciare la coerenza, le prestazioni
          di lettura e scrittura.
    \item Versionamento dei dati: permette la gestione di più versioni di oggetti nel
          sistema, richiedendo la risoluzione delle versioni duplici da parte
          dell'applicazione o dell'utente.
\end{itemize}

\textbf{Principali caratteristiche di Dynamo}
\begin{itemize}
    \item Hashing consistente: assegna le chiavi ai nodi in modo efficiente, consentendo
          l'aggiunta o la rimozione di nodi con minimo riassestamento.
    \item Coerenza regolabile: permette di scegliere tra coerenza, prestazioni di lettura
          e scrittura.
    \item Versionamento dei dati: gestisce più versioni di oggetti, richiedendo la
          risoluzione delle versioni duplici da parte dell'applicazione o dell'utente.
\end{itemize}

\begin{domanda}
    Qual e' la differenza tra SQL e NOSQL
\end{domanda}

Nei database NOSQL si perde la strictness del modello relazionale e si parla piu' 
di \textbf{modello a documenti}. Questo significa che i dati sono salvati con una
struttura diversa tra loro, non obbligatoriamente. Il formato con cui si 
salvano e' spesso \textbf{xml} o \textbf{json}.

Ogni documento possiamo dire che rappresenti una riga di un database relazionale. All'interno
del documento ci potrebbero essere anche documenti innestati e liste.

\newpage
\section{Domande: Google Stack e Hadoop}

\begin{domanda}
    Come funziona il file system di Google
\end{domanda}

Il GFS o Google File System si basa sul concetto di avere costantemente i loro
servizi attivi, avendo a disposizione un \textbf{numero enorme} di macchine e
server che garantiscono l'\textbf{availability} costante.

La consistenza non e' un qualcosa che viene garantito, ma viene garantito una
\textbf{fault tolerance} costante. Anche perche' avendo $1.000.000$ di
macchine, il fatto che 1 non funzioni diventa la normalita'. Ci si aspetta di
avere anche dei sistemi che ripristino il sistmea quando questo accade.

\begin{itemize}
    \item Faul tolerance: tante macchine che permettono di avere un'espansione
          orizzontalme
    \item Banda sostenuta: garantire sempre il servizio a discapito della latenza
    \item Atomicita' con minimi problemi di sincronizzazione
    \item \textbf{Availability}
\end{itemize}

Si basa su assunzioni di dover gestire tanti file dell'ordine di GB, di avere
\textit{poche letture random e tante letture a batch}.

\begin{domanda}
    Qual e' la struttura del GFS
\end{domanda}

\begin{itemize}
    \item Tanti client: accedono al servizio
    \item Un Master: Gestisce i metadati e comunica con i chunkserver
    \item Tanti chunckserver: Salva i dati e sul disco locale. Spesso ogni chunck viene
          duplicato per garantire la fault tolerance e availability.
\end{itemize}

Diciamo che la procedura di comunicazione dell'applicazione e' che:
\begin{enumerate}
    \item Il client chiede al master quale chunckserver contattare
    \item Il client contatta il chunck server
    \item Vengono effettuate le operazioni sui chunk
    \item Il master controlla lo stato dei chunckserver e aggiorna i metadati
\end{enumerate}

\begin{domanda}
    Come funziona il paradigma MapReduce?
\end{domanda}

\textbf{Nota}: IN un programma parallelo quali sono i maggiori bottleneck? \textbf{I LOCK}

Il paradigma si basa su due fasi:
\begin{enumerate}
    \item Map: I dati vengono partizionati in chunks che saranno poi gestiti in parallelo \item Reduce: Si combina l'output dei vari mappers per avere il risultato finale
\end{enumerate}

Da notare che viene effetuato \textbf{con bruteforce} e il codice fa parecchio
schifo mi sa.

%Esempio di word count
\begin{lstlisting}
    map(String key, String value):
        // key: document name
        // value: document contents
        for each word w in value:
            EmitIntermediate(w, "1");

    reduce(String key, Iterator values):
        // key: a word
        // values: a list of counts
        int result = 0;
        for each v in values:
            result += ParseInt(v);
        Emit(AsString(result));
\end{lstlisting}

\begin{domanda}
    Come funziona BigTable?
\end{domanda}

BigTable e' un database distribuito che si basa su GFS e MapReduce. E' un
database \textbf{non relazionale} che si basa su una struttura di
\textbf{colonne} e \textbf{righe}.

\begin{quote}
    Sparsa distribuita persistente multi dimensionale ordinata MAPPA

    (riga: stringa, colonna:stringa, tempo: int64 ) $\rightarrow$ stringa
\end{quote}

Le pagine web sono salvate con URL al contrario perche per distribuire i dati
in modo uniforme sui server, BigTable deve utilizzare una chiave di
partizionamento. In questo caso, la chiave di partizionamento è l'URL della
pagina web.

Per evitare che tutti i dati vengano memorizzati su un unico server, BigTable
utilizza una funzione di hash per trasformare l'URL in una chiave di
partizionamento. Tuttavia, se l'URL fosse utilizzato come chiave di
partizionamento così com'è, ci sarebbe il rischio che tutti i dati relativi a
un singolo sito web finiscano sullo stesso server.

Per evitare questo problema, Google salva gli URL al contrario. In questo modo,
l'hash della chiave di partizionamento viene distribuito in modo più uniforme
sui server, migliorando le prestazioni del database.

\begin{itemize}
    \item Righe: sono stringhe arbitrarie che identificano la riga
    \item Colonne: sono stringhe arbitrarie che identificano la colonna. Si chiamano
          famiglie. Spesso i dati della stessa famiglia sono dello stesso tipo.
    \item Timestamp: Viene usato per poter avere una versione dei dati e poter fare
          rollback
\end{itemize}

\begin{domanda}
    Implementazione di BigTable
\end{domanda}

Ha 3 elementi principali.

Il client che usa le API per comunicare. Un \textbf{Master} che gestisce i
tablet servers e garantisce il load balancing. \textbf{I tablet servers} che
gestiscono un insieme di tablets.

I tablet servers gestiscono il partizionamento dei dati e le richieste in
lettura e scrittura.

Notare che i\textbf{ dati del client non passano dal master ma dai tablet
    servers}

\begin{domanda}
    Come funziona Hadoop
\end{domanda}

Hadoop e' un framework open source che permette di gestire grandi quantita' di
dati in modo distribuito. Si basa su HDFS e MapReduce.

Lo \textbf{schema on read} permette di non dover progettare uno schema per i
dati. Lo schema dipende dai dati che si andranno a gestire. Anche questo si
basa sull'aggiungere hardware comodo ed economico per scalare.

\begin{itemize}
    \item Scale out
    \item Chiave VALORE
    \item Functional Programming
    \item Offline batch (non realtime)
\end{itemize}

\begin{domanda}
    La struttura di Hadoop
\end{domanda}

\begin{itemize}
    \item NameNode: gestisce i metadati e i file system (\textbf{UNO E UNICO})
    \item DataNode: ogni macchina slave avra' un datanode che gestisce i dati.
          (\textbf{SONO TANTI})
    \item DataNode secondario: Usato come backuo diciamo (\textbf{tipicamente uno})
    \item JobTracker: gestisce i job e le richieste dei client (\textbf{Uno solo,
              master})
    \item TaskTracker: gestisce i task sui vari slave node(\textbf{tanti, slave})
\end{itemize}

%create a fig
\usetikzlibrary{positioning}
\begin{tikzpicture}

    % Client
    \node [draw, rectangle] (client) {Client (Hadoop)};
    \node [below=4cm of client] (dummy) {}; % Dummy node for spacing

    % Resource Manager
    \node [draw, rectangle, right=5cm of client] (rm) {Resource Manager};
    \node [below=2cm of rm] (dummy2) {}; % Dummy node for spacing

    % Nodi
    \node [draw, rectangle, above right=1cm and 2cm of dummy] (node1) {Nodo 1};
    \node [draw, rectangle, right=2cm of node1] (node2) {Nodo 2};
    \node [draw, rectangle, right=2cm of node2] (node3) {Nodo 3};

    % Freccia da Client a Resource Manager
    \draw[->] (client.east) -- (rm.west) node[midway, above] {Richiesta di Lavoro};

    % Freccia da Resource Manager a Nodi
    \draw[->] (rm.south) -- (node1.north) node[midway, left] {Task};
    \draw[->] (rm.south) -- (node2.north) node[midway, left] {Task};
    \draw[->] (rm.south) -- (node3.north) node[midway, left] {Task};

\end{tikzpicture}

\newpage


\section{Domande: Column and in Memory database}

\begin{domanda}
    Per quale motivo sono nati i database a grafo?
\end{domanda}

I database relazionali immagazzinavano dati e le relazioni che ci sono tra
loro. La controparte di queste relazioni e' il \textbf{grafo}. Ma per modellare
un grafo non era facile con i sistemi di database relazionali. Ancora peggio i
database non relazionali non riuscivano a modellare per bene i grafi,
sopratutto quando si parla di relazioni tra oggetti (non ci sono i join).

Comunque in generale per questa parte non c'e' molto da dire. Cioe' provarono a
costruire dei database basati sui grafi. Avevano alcune proprieta' decenti come
la \textbf{index free adjacency} che permetteva di muoversi all'interno del
grafo con senza dover cercare gli indici. Comunque c'erano dei problemi.

C'erano alcuni engine che permettevano di calcoalre questi grafi:
\begin{itemize}
    \item Apache Giraph
    \item GraphX
    \item Titan
\end{itemize}

\begin{domanda}
    Cos'e RDF
\end{domanda}

RDF significa \textbf{resource description framework RDF}. E' un framework per
rappresentare le informazioni sul web.
\begin{equation}
    entity:attribute:value
\end{equation}

Utilizzala sintassi XML e il linguaggio SPARQL
\begin{lstlisting}[language=SPARQL, caption={Esempio di query SPARQL}]
    SELECT ?object
        FROM <http://example.org>
        WHERE {
            <http://example.org> <http://example.org/property> ?object
        }
\end{lstlisting}

\begin{domanda}
    Differenza tra OLTP e OLAP
\end{domanda}

OLTP sta per \textbf{online transaction processing} e OLAP sta per
\textbf{online analytical processing}. OLTP e' un sistema che permette di
gestire le transazioni ed e' comodo sopratutto quando si ha un carico elevato
di \textbf{transizioni in scrittura}. Gli ecommerce ad esempio sfruttano molto
i sistemi OLTP. I sistemi OLTP hanno uno schema \textbf{row oriented}.

I sistemi OLAP sono piu' utili in caso di sistemi che hanno un elevato numero
di richieste in \textbf{lettura} come ad esempio i software che devono
calcolare spesso medie e misure, come ad esempio i software di business
intelligence e data mining. I sistemi OLAP hanno uno schema \textbf{column
    oriented}.

\begin{domanda}
    Per cosa e' buono avere database row o column
\end{domanda}

Ad esempio, poniamo il caso di avere un database con 10 colonne e a noi
interessa la colonna 'eta'

\begin{table}[H]
    \centering
    \begin{tabular}{|c|c|c|c|c|}
        \hline
        id & nome  & cognome & eta & indirizzo  \\ \hline
        1  & Mario & Rossi   & 20  & Via Roma   \\ \hline
        2  & Luca  & Bianchi & 30  & Via Milano \\ \hline
        3  & Marco & Verdi   & 40  & Via Napoli \\ \hline
    \end{tabular}
\end{table}

Se a noi interessa ad esempio la media delle eta', con un database row oriented
dobbiamo scorrere tutte le righe e prendere la colonna eta' e calcolare la
media. Con un database column oriented invece, abbiamo gia' la colonna eta' e
quindi possiamo calcolare la media direttamente.

\begin{domanda}
    A cosa serviva C-Store Storage e come funzionava
\end{domanda}

C-store sfruttava la ridondanza degli elementi con le proiezioni. Una
proiezione e' una ripetizione di alcune colonne che sono accesse frequentemente
e salvate sul disco. Questo aumentava le performance.

Aveva una pesante compressione delle colonne. Cioe, le colonne venivano
comprese e salvate sul disco. Questo permetteva di avere un accesso piu'
veloce.

\begin{itemize}
    \item C-Store storage utilizza la compressione delle colonne per migliorare le
          prestazioni.
    \item I dati sono archiviati in ordine ordinato e le colonne sono pesantemente
          compresse.
    \item Il sistema utilizza un Writeable (Delta) Store e un Read-Optimized Store per
          gestire le operazioni di scrittura e lettura.
    \item C-Store Storage utilizza la tecnica K-safety per garantire l'affidabilità dei
          dati.
\end{itemize}

\begin{domanda}
    Per quale motivo i sistemi heavy write non funzioano bene su ssd?
\end{domanda}

La risposta e' molto semplice: Per scrivere su SSD bisogna cancellare e
riscrivere. Quindi se si ha un sistema che scrive molto, si avranno molti cicli
di scrittura e cancellazione. Questo ha un effetto di bottleneck sul sistema.

\begin{domanda}
    Database cacheless
\end{domanda}

I database in memoria principale (Main Memory DB) sono progettati per
funzionare senza cache, ovvero senza la necessità di memorizzare
temporaneamente i dati su memoria cache per minimizzare l'accesso ai dati su
disco.

A differenza dei database tradizionali basati su disco, i database in memoria
principale memorizzano tutti i dati direttamente in memoria principale, senza
la necessità di accedere ai dati su disco. In questo modo, l'accesso ai dati è
molto più veloce e non è necessario utilizzare una cache per minimizzare
l'accesso ai dati su disco.

In altre parole, la cache è inutile in un database in memoria principale perché
tutti i dati sono già memorizzati in memoria principale. Non c'è bisogno di
memorizzare temporaneamente i dati su cache perché non c'è bisogno di
minimizzare l'accesso ai dati su disco

\begin{domanda}
    Cosa e' Berkeley Analytics Data Stack
\end{domanda}

BDAS è un insieme di strumenti open source per l'analisi dei dati formato da:
\begin{itemize}
    \item Spark: un framework per l'elaborazione distribuita
    \item Mesos: un sistema operativo per i data center (Gestisce i cluster)
    \item Tachyon: un sistema di file distribuito
\end{itemize}

\begin{domanda}
    Cosa e' SPARK
\end{domanda}

Spark e' un framework che permette di lavorare sui dati in modo distribuito.

Tratta i dati come RDD, cioe' \textit{resilient distributed datasets}. Questi
RDD sono immutabili e sono distribuiti sui nodi del cluster. Spark permette di
fare operazioni su questi RDD in modo parallelo.

Non abbiamo visto ancora niente di Spark, quindi non posso dire altro ad essere onesto.


\section{Domande: MapReduce Introduction}
\begin{domanda}
    Per quale motivo Google ha inventato Map Reduce?
\end{domanda}

Perche' aveva bisogno di un sistema che permettesse il processamento di un
quantitativo enorme di file tra diverse macchine, in modo parallelo, con una
tolleranza ai guasti e senza usare codici troppo complessi. Per questo motivo,
nasce questo sistema che permette di fare tutto cio'. Ah questo andava fatto
pero' su diverse macchine che non costassero un patrimonio, quindi doveva
essere anche economico.

\begin{quote}
    I programmatori che non erano esperti di programmazione distribuita dovevano essere in grado di effettuare operazioni su sistema distribuiti a larga scala.
\end{quote}

\begin{domanda}
    Cos'e' MapReduce
\end{domanda}

MapReduce e' un modello di programmazione che soddisfa i seguenti punti:
\begin{itemize}
    \item Programma per processare grandi insiemi di dati 
    \item Partizionare i dati di input 
    \item Schedulare l'esecuzione su un cluster di macchine
    \item Gestire i guasti
    \item Gestire la comunicazione tra le macchine
\end{itemize}

\begin{domanda}
    Quali sono i componenti principali di MapReduce
\end{domanda}

I principali componenti di MapReduce sono due funzioni:
\begin{itemize}
    \item Funzione mapping
    \item Funzione reducing
\end{itemize}

\[
  map(k1,v1) \rightarrow list(k2,v2)
\]
Prende in \textbf{input} una coppia chiave valore e restituisce in \textbf{output} una lista di
coppie chiave valore.
\[
  reduce(k2,list(v2)) \rightarrow list(v2)
\]
Prende in \textbf{input} una chiave e una lista di valori e restituisce in \textbf{output} una lista di
valori. (Oppure una coppia chiave valore)

Praticamente il workflow e':
\begin{enumerate}
    \item Map: Si prende un file e lo si divide in blocchi
    \item Map: Si applica la funzione map a tutti i blocchi
    \item I risultati vengono salvati sulle macchine locali 
    \item Reduce: Si prendono i risultati e si applica la funzione reduce
    \item Reduce: Si salvano i risultati finali sul GFS o HDFS
\end{enumerate}

\begin{lstlisting}[language=java][caption=MapReduce in Java]
    map(String key, String value):
        // key: document name
        // value: document contents
        for each word w in value:
            EmitIntermediate(w, "1");
    
    reduce(String key, Iterator values):
        // key: a word
        // values: a list of counts
        int result = 0;
        for each v in values:
            result += ParseInt(v);
        Emit(AsString(result));
\end{lstlisting}


\begin{domanda}
    Cosa succede in questo caso se il file per word count ha solamente 1 parola ripetuta 1000 volte?
\end{domanda}

In questo caso, il lavoro del mapping sara' super parallelo, ma il lavoro 
del reducing sara' sequenziale, perche' tutti i risultati del mapping
verranno mandati al reducer che si occupera' di fare il reduce.

\begin{domanda}
    La fase di shuffle cosa fa?
\end{domanda}

La fase di shuffle e' la fase che si occupa di \textbf{mandare i risultati del mapping
al reducer corretto}. In questo caso, il reducer corretto e' quello che si occupa
di fare il reduce della parola che e' stata mappata.


\begin{domanda}
    Workflow di ma reduce nel dettaglio
\end{domanda}

\begin{enumerate}
    \item Viene splittato il file di input in diversi file 
    \item Si passa il programma al cluster di macchine
    \item La macchina master divide il lavoro tra i worker
    \item Ci sono $M$ task di mapping e $R$ task di reducing
    \item I worker leggono il contenuto del file di input assegnato dal master
    \begin{itemize}
        \item Per ogni coppia chiave valore, viene applicata la funzione map
        \item Le coppie vengono salvate nella memoria buffer
        \item I risultati vengono salvati in memoria locale periodicamente leggendo dalla memoria buffer in $R$ regioni.
    \end{itemize}
    \item Il master viene informato di dove sono queste $R$ regioni
    \item Manda le informazioni sulla regione ai reducer
    \item I reducer leggono i risultati dal disco locale tramite procedure remote 
    \begin{itemize}
        \item I risultati vengono ordinati per chiave
        \item I risultati vengono passati alla funzione reduce
        \item I risultati vengono aggiunti alla fine del file output (\textit{appending})
    \end{itemize}
    \item Tutte le macchine comunicano con il master che hanno terminato il lavoro
    \item Il master manda il segnale di terminazione al client
    \item Il file output e' disponibile per il client
\end{enumerate}

\begin{domanda}
    Come fa MapReduce ad essere fault tolerant?
\end{domanda}

Dobbiamo dire innanzitutto che abbiamo due tipi di fallimenti:
\begin{itemize}
    \item Worker Failure
    \item Master Failure
\end{itemize}

\textbf{Worker Failure}: Il master tenta di tanto in tanto di pingare il worker. Se esso non risponde entro un certo limite di tempo, allora il master \textbf{segna il worker come inattivo}, e quindi questo implica che:
\begin{itemize}
    \item Le task che dovevano essere eseguite dal worker e anche quelle gia completate vengono riassegnate ad altri worker
    \item Le task gia completate vengono eseguite dinuovo da altri worker: questo perche' ogni worker scrive sulla memoria locale, e quei file sono ormai perduti e non piu accessibili
    \item Discorso diverso se il worker era un \textbf{reducer}: Se il reducer ha gia' completato il suo lavoro ma non risponde, la sua task non va riassegnata perche' ha gia' inserito l'output della funzione reduce nel file di output corretto, che e' disponbiile sul filesystem.
\end{itemize}

\textbf{Master Failure}: Se il master fallisce gli facciamo il funerale. Si ferma la computazione di MapReduce.

\begin{domanda}
    Spiega la locality di MapReduce
\end{domanda}

In MapReduce si vuole risparmiare banda piu' che si puo'. 
Per questo motivo, i file hanno un fattore di replicazione che dipende dalla configurazione del cluster.
I dati iniziali vengono salvati su dischi locali delle macchine e in multipla copia.

Quando viene \textbf{assegnato un task di map}, si tende ad assegnarlo alle macchine che hanno i file \textit{vicini} oppure gia' salvati all'interno della propria memoria come copia. Questo minimizza il consumo di banda.

\begin{domanda}
    Spiega la task granularity di MapReduce
\end{domanda}

Diciamo che MapReduce tende a scegliere un numero di Mappers $M$ 
tale che ogni compito di mapping richieda circa 16-64 MB di input.

Sceglie $R$ un piccolo multiplo delle macchine del cluster, per bilanciarne il carico. 

Sceglie il numero di $M$ e $R$ in modo che siano molto piu' grandi del numero di macchine nel cluster.

\begin{domanda}
    Spiega backup tasks in MapReduce
\end{domanda}

Allevia il problema dei \textbf{ritardatari} o stragglers. 
Cioe', il master schedula delle esecuzioni backup delle task in corso. La task viene segnata come completa 
indipendentemente dal fatto che la task originale sia completata o meno. Cioe' la cosa che
gli interessa di piu' e' che la task sia completata da qualche worker. 

\newpage
\section{Domande: MapReduce Hadoop Basics}

\begin{domanda}
    I data types di Hadoop
\end{domanda}

Hadoop non ammette qualsiasi tipo di dati, poiche' per essere mandati i valori,
devono essere \textbf{serializzati}.

\begin{definition}(Serializzazione)
    La serializzazione e' il processo di conversione di un oggetto
    in un formato che puo' essere memorizzato (ad esempio, in un file
    o in un buffer di memoria) o trasmesso (ad esempio, attraverso
    una connessione di rete) e ripristinato in un oggetto con le
    stesse caratteristiche dell'originale.
\end{definition}

Ci sono diverse interfacce che devono essere implementate per le
\textbf{chiavi} e per i \textbf{valori}.

\begin{itemize}
    \item Valori: I valori devono implementare l'interfaccia \textbf{Writable}
    \item Chiavi: Le chiavi devono implementare l'interfaccia
          \textbf{WritableComparable$\langle T \rangle$}. C'e' bisogno di questo
          comparable perche' successivamente si andra' a fare un sort sulle chiavi.
\end{itemize}

%tabella con i vari tipi  di clase e descrizione 
\begin{table}[H]
    \centering
    \begin{tabular}{|c|c|}
        \hline
        \textbf{Classe}          & \textbf{Descrizione}                                 \\
        \hline
        \textbf{BooleanWritable} & Valore booleano                                      \\
        \hline
        \textbf{ByteWritable}    & Valore byte                                          \\
        \hline
        \textbf{DoubleWritable}  & Valore double                                        \\
        \hline
        \textbf{FloatWritable}   & Valore float                                         \\
        \hline
        \textbf{IntWritable}     & Valore intero                                        \\
        \hline
        \textbf{LongWritable}    & Valore long                                          \\
        \hline
        \textbf{Text}            & Valore stringa formato UTF8                          \\
        \hline
        \textbf{NullWritable}    & Valore nullo usato quando la chiave non e' richiesta \\
        \hline
    \end{tabular}
    \caption{Tipi di classe di Hadoop}
    \label{tab:my_label}
\end{table}

\begin{domanda}
    Si possono creare classi personalizzate?
\end{domanda}

La risposta e; \textbf{si}. Bisogna fare l'override dei metodi:
\begin{itemize}
    \item readFields(DataInput in)
    \item write(DataOutput out)
    \item compareTo(T o)
\end{itemize}

\begin{domanda}
    Cosa fa la classe MAPPER?
\end{domanda}

La classer Mapper e' una classe astratta che deve essere estesa per
implementare il metodo \textbf{map}.

public class \textbf{Mapper}$\langle KEYIN,VALUEIN,KEYOUT,VALUEOUT \rangle$

%aggiungi linea di codice
\begin{lstlisting}[language=Java]
    public class TokenCounterMapper extends Mapper<Object, Text, Text, IntWritable>
\end{lstlisting}

Esistono gia' dei \textbf{mappers standard}.

\begin{table}[H]
    \begin{center}
        \begin{tabular}{|c|c|}
            \hline
            \textbf{Classe}           & \textbf{Descrizione}                                                      \\
            \hline
            \textbf{IdentityMapper}   & Mapper che emette le coppie chiave-valore in input senza modificarle.     \\
            \hline
            \textbf{InvertMapper}     & Mapper che inverte le coppie chiave-valore in input.                      \\
            \hline
            \textbf{RegexMapper}      & Mapper che emette coppie chiave-valore in base a un'espressione regolare. \\
            \hline
            \textbf{TokenCountMapper} & Mapper che emette coppie chiave-valore in base a un'espressione regolare. \\
            \hline
        \end{tabular}
    \end{center}
\end{table}

\begin{domanda}
    Cosa fa la classe REDUCER?
\end{domanda}

La classe reducer e' una classe astratta che deve essere estesa per
implementare il metodo \textbf{reduce}.

Dato un insieme di dati che condividono la \textbf{stessa chiave}, riduce
l'insieme ad un numero inferiore di elementi.

\begin{itemize}
    \item Shuffle and sort: Prende i valori dai mapper con delle \textbf{richieste http}
          ed effettua un merge sort sulle chiavi
    \item Sort Secondario: Estende la chiave con la chiave seocndaria e definisce un
          comparatore di gruppo
\end{itemize}

public clas Reducer$<KEYIN,VALUEIN,KEYOUT,VALUEOUT>$

\begin{lstlisting}[language=Java]
    public class IntSumReducer<Key> extends Reducer<Key,
                            IntWritable, Key,IntWritable>
\end{lstlisting}

Ci sono anche reducer standard:
\begin{table}[H]
    \begin{center}
        \begin{tabular}{|c|c|}
            \hline
            \textbf{IdentityReducer} & Reducer che emette le coppie chiave-valore in input senza modificarle. \\
            \hline
            \textbf{LongSumReducer}  & Reducer che somma i valori in input.                                   \\
            \hline
        \end{tabular}
    \end{center}
\end{table}

\begin{domanda}
    Cosa sono i Combiner?
\end{domanda}

I \textbf{Combiner} non sono altro che dei \textbf{reducer locali}. In parole
povere, sono dei worker che diminuiscono il lavoro a carico dei
\textbf{reducer}. \textit{Implementa la stessa interfaccia} \textbf{Reducer}
Quando ci sono molti dati, possono diminuire il traffico di rete. Non e' sempre
possibile applicarlo, attenzione!

Come fanno a sapere dove prendere i dati, pero'? Gli viene detto dal
\textbf{master.}

\begin{domanda}
    Cosa sono i Partitioner?
\end{domanda}

I partitioner determinano dove vanno mandate le coppie chiave-valore che
vengono sputate fuori dall'output dei mapper. \textbf{Attenzione}: I
partitioner esistono solamente quando \textbf{c'e' piu' di un reducer!}

Di default, c'e' \textbf{HashParitioner} come classe di Hadoop.

\begin{domanda}
    InputFormat classe e interfaccia
\end{domanda}

L'interfaccia definisce come l'input viene diviso e letto in Hadoop. Esistono
diverse implementazioni e puoi fartene quante ne vuoi .

La classe descrive come specificare i dati di input per un lavoro di MapReduce.
Divide i file di input in \textbf{InputSplits} che vengono assegnati ai Mapper
individuali. Viene utilizzato un'implementazione chiamata \textbf{RecordReader}
per estrarre i record di input dagli InputSplits.

\begin{table}[H]
    \begin{center}
        \begin{tabular}{|c|}
            \hline
            \textbf{Classe}                  \\
            \hline
            \textbf{TextInputFormat}         \\
            \hline
            \textbf{KeyValueTextInputFormat} \\
            \hline
            \textbf{SequenceFileInputFormat} \\
            \hline
            \textbf{NLineInputFormat}        \\
            \hline
        \end{tabular}
    \end{center}
\end{table}

\begin{domanda}
    Come si crea un custom input format
\end{domanda}

Per farlo, la funzione InputFormat deve identificare tutti i file usati come
dati di input e dividerli in splits.

Bisogna fornire un oggetto \textit{RecordReader} dove bisogna iterare sui
record e restituire la chiave e il valore.

\begin{lstlisting}[language=Java]
    public classe CustomRecordReader extends RecordReader {
        @Override
        public void initialize(InputSplit split, TaskAttemptContext context) throws IOException, InterruptedException {
            //inizializza il record reader
        }

        @Override
        public boolean nextKeyValue() throws IOException, InterruptedException {
            //itera sui record e restituisci la chiave e il valore
        }

        @Override
        public Object getCurrentKey() throws IOException, InterruptedException {
            //restituisci la chiave corrente
        }

        @Override
        public Object getCurrentValue() throws IOException, InterruptedException {
            //restituisci il valore corrente
        }

        @Override
        public float getProgress() throws IOException, InterruptedException {
            //restituisci il progresso
        }

        @Override
        public void close() throws IOException {
            //chiudi il record reader
        }
    }
\end{lstlisting}

\begin{domanda}
    Formati di output
\end{domanda}

Gli output file di mapreduce usano la classe \textbf{OutputFormat}. Non ha
nesusno split e ogni reducer scrive un singolo file di output. L'oggetto
\textbf{RecordWriter} formatta l'output.

\begin{table}
    \begin{center}
        \begin{tabular}{|c|c|}
            \hline
            \textbf{Classe}                   & \textbf{Descrizione}                                                                                 \\
            \hline
            \textbf{TextOutputFormat}         & Scrive i file di testo.                                                                              \\
            \hline
            \textbf{SequenceFileOutputFormat} & Custom separator \\
            \hline
            \textbf{NullOutputFormat}         & Non scrive alcun output.                                                                             \\
            \hline
        \end{tabular}
    \end{center}
\end{table}

\begin{domanda}
    La classe Configuration
\end{domanda}

Questa classe permette l'accesso ai parametri di configurazione.
\begin{lstlisting}[language=java]
    public class Configuration extends Object 
    implements iterable<Map.Entry<String,String>>, Writable
\end{lstlisting}

\begin{domanda}
    Cosa sono i job?
\end{domanda}

I job sono le unita' di lavoro di Hadoop. L'utente crea l'applicazione e descrive come deve essere eseguita tramite i \textbf{Job}. Manda poi i job ad essere eseguiti.

\begin{lstlisting}[language=Java]
    public class Job extends JobContextImpl
     implements JobContext, AutoCloseable
\end{lstlisting}

Dei job si possono modificare:
\begin{itemize}
    \item Il nome 
    \item Il jar che contiene il codice da eseguire 
    \item Configurare input e output
    \item Configurare mapper(s) e reducer(s)
\end{itemize}

Per eseguire e controllare l'esecuzione: \textit{job.waitForCompletion(true)}

\begin{lstlisting}[language=Java]
    //Create il job 
    Job job = Job.getInstance();
    job.setJarByClass(MyJob.class);

    //Settare alcuni parametri
    job.setName("JobProva");

    job.setInputPath(new Path("input"));
    job.setOutputPath(new Path("output"));
    
    job.setMapperClass(MyJob.MyMapper.class);
    job.setReducerClass(MyJob.MyReducer.class);

    //Mandare il job e pullare i risultati fino alla fine del job
    job.waitForCompletion(true);
\end{lstlisting}


\end{document}