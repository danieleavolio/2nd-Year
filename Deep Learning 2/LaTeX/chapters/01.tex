\section{Deep Learning 101}

In questo corso affronteremo diverse tematiche, il che può sembrare assurdo se ci si pensa.

\begin{itemize}
  \item Classificazione (binaria, cioè sì o no)
  \item Multi-class classification (non più binaria)
  \item Regressione (il guessing viene fatto su un valore numerico)
  \item Gestione di immagini e riconoscimento
  \item Serie numeriche (predizioni di mercato e trend)
  \item Classificazione di testi
\end{itemize}

\subsection{Architetture e strumenti nel deep learning}

\begin{itemize}
  \item Autoencoder
    \begin{itemize}
      \item Tutti i possibili tipi
      \item Qui si fa anche \textbf{Clustering} e \textbf{Anomaly detection}
    \end{itemize}
  \item Architetture generative
    \begin{itemize}
      \item Tutti i possibili tipi
    \end{itemize}
  \item XAI: Explainable AI
\end{itemize}

\subsection{Libri utili}
\begin{itemize}
  \item "Deep Learning in Python"
  \item "Tensorflow tutorial"
\end{itemize}

\subsection{Strumenti che useremo}
\begin{itemize}
  \item Tensorflow
    \begin{itemize}
      \item High-level più di altri
    \end{itemize}
  \item Keras
    \begin{itemize}
      \item High-level API basato su Tensorflow
      \item Ci saranno cose che non possiamo fare con Keras perché è troppo ad alto livello
    \end{itemize}
\end{itemize}

\subsection{Schema generale di un problema di deep learning}

Abbiamo delle coppie $(x_0,y_0), (x_1,y_1), (x_2,y_2), \ldots, (x_n, y_n)$ dove $x_i$ è un vettore di features e $y_i$ è un valore numerico (regressione) o una classe (classificazione).

$$y_i = f(x_i)$$

Non conosciamo la funzione $f$, quindi dobbiamo impararla.

$$y = \alpha x + \beta$$

Con una rete neurale puoi approssimare praticamente qualsiasi funzione.

\textit{Una rete neurale permette di collegare un input di dati a una funzione di output.}

Abbiamo diversi tipi di reti neurali a seconda del tipo di problema che vogliamo risolvere. È importante essere in grado di selezionare l'architettura giusta per risolvere il problema.

Definiamo alcuni concetti che useremo:
\begin{itemize}
  \item $N$: rete neurale
  \item $w$: valori dei pesi della rete neurale
  \item $f$: funzione di output della rete neurale
\end{itemize}

$$ f \in N(w)$$

\subsection{Perché si usa il termine "Tensore"?}

Un \textit{tensore} non è altro che una matrice.
\begin{itemize}
  \item 0D tensor: scalar
  \item 1D tensor: vector
  \item 2D tensor: matrix
  \item 3D tensor: tensor
\end{itemize}

\subsection{AI vs DL}

\begin{itemize}
  \item AI: è un ampio insieme di tecniche per risolvere problemi che richiedono "intelligenza".
    \begin{itemize}
      \item Esempio: Stockfish, un programma che gioca a scacchi.
    \end{itemize}
  \item DL: È un sottoinsieme di AI che si concentra sull'\textit{astrazione}.
    \begin{itemize}
      \item L'astrazione consiste nel fornire una funzione che traduce dati di input in dati di output senza conoscere la funzione stessa.
      \item È un approccio induttivo: fornisci un input e ti aspetti un output, senza conoscere la funzione che li collega.
      \item Questo è completamente diverso dall'AI basata sulla logica.
    \end{itemize}
\end{itemize}
