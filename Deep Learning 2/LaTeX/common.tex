
% Packages for code, figures, and automata
\usepackage{listings} % For code listings
\usepackage{graphicx} % For including figures
\usepackage{tikz}     % For drawing automata
\usepackage{multirow}
\usepackage{array}
\usepackage{amssymb} % Pacchetto per simboli matematici
\usepackage{float}

\usepackage{pgfplots}
\usepackage{amsmath}

%colorize lstlisting with language
\usepackage{xcolor}

\usepackage{titlesec}

%package for theorem
\usepackage{amsthm}

%import all important packages 

% Configurazione degli stili per tutti i linguaggi
\lstset{
    basicstyle=\ttfamily,
    keywordstyle=\color{blue},
    commentstyle=\color{purple},
    stringstyle=\color{red},
    % Altre opzioni
    breaklines=true, showstringspaces=false,
    emph={label},
    emphstyle={\color{custompurple}},
    escapeinside={(*}{*)}
    }

% Stile globale per tutti i linguaggi
\lstdefinestyle{mystyle}{
    backgroundcolor=\color{white},
    commentstyle=\color{purple},
    keywordstyle=\color{blue},
    numberstyle=\tiny\color{gray},
    stringstyle=\color{red},
    basicstyle=\ttfamily\footnotesize,
    breakatwhitespace=false,
    breaklines=true,
    captionpos=b,
    keepspaces=true,
    numbers=left,
    numbersep=5pt,
    showspaces=false,
    showstringspaces=false,
    showtabs=false,
    tabsize=2
}

%package for theorem
\usepackage{amsthm}

\usepackage{pgfplots}
\usepackage{amsmath}



% Impostazioni per tutti i linguaggi
\lstset{style=mystyle}
% Definizione di simboli per subsection e subsubsection
\newcommand{\subsecsymbol}{\textcolor{custompurple}{\rule[0pt]{10pt}{10pt}\hspace{10pt}}}
\newcommand{\subsubsecsymbol}{\textcolor{custompurple}{\textbf{$\blacklozenge$}\hspace{4pt}}}

\titleformat{\section}[block]
  {\Huge\bfseries}
  {\llap{\textcolor{custompurple}{\rule[-4pt]{10pt}{18pt}\hspace{10pt}}\thesection\hskip 12pt}}
  {0pt}
  {}
% Definizione di uno stile per \subsection
\titleformat{\subsection}[block]
  {\Large\bfseries\color{black}}
  {\llap{\subsecsymbol}\thesubsection\hskip 12pt}
  {0pt}
  {}

% Definizione di uno stile per \subsubsection
\titleformat{\subsubsection}[block]
  {\large\bfseries\color{black}}
  {\llap{\subsubsecsymbol}\thesubsubsection\hskip 12pt}
  {0pt}
  {}

\newtheorem{definition}{Definizione}[section] % Crea un nuovo ambiente "definition"
\newtheorem{theorem}{Teorema}[section] % Crea un nuovo ambiente "theorem"
\newtheorem{corollary}{Corollario}[theorem] % Crea un nuovo ambiente "corollary"
\newtheorem{lemma}[theorem]{Lemma} % Crea un nuovo ambiente "lemma"
\newtheorem{domanda}{Domanda}[section] % Crea un nuovo ambiente "domanda"
\newtheorem{osservazione}{Osservazione}[section] % Crea un nuovo ambiente "osservazione"
\newtheorem{esempio}{Esempio}[section] % Crea un nuovo ambiente "esempio"
\usepackage{listings}



%make link clickable
\usepackage{hyperref}
\usepackage{pgfplots}

%use asmath
\usepackage{amsmath}

\usepackage{fancyhdr}
\pagestyle{fancy}
\fancyhf{}
\fancyhead[R]{\nouppercase{\rightmark}}
\fancyfoot[C]{\thepage}

\usepackage{listings}
\usepackage{tabularx}

%color link orange
\hypersetup{
    colorlinks=true,
    linkcolor=custompurple,
    filecolor=magenta,
    urlcolor=cyan,
}

\definecolor{custompurple}{HTML}{8b3fff}
